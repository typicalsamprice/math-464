% -*- compile-command: "latexmk -pdf document.tex" -*-
\documentclass[11pt]{article}

\author{Sam Price}
\title{Heine-Borel}

\newif\ifprinted%
\printedtrue%
% Most taken from: https://github.com/SeniorMars/dotfiles/blob/main/latex_template/preamble.tex
\usepackage{geometry}

\usepackage[english]{babel}
\usepackage[T1]{fontenc}
\usepackage[utf8]{inputenc}
\usepackage{palatino}

% Use the command \doublespacing if needed
\usepackage{setspace}

\geometry{a4paper, margin=1in}

\usepackage{mathtools}
\usepackage{amssymb,amsfonts,amsthm,amsmath}

\usepackage{xfrac}
\usepackage[makeroom]{cancel}

% Use \begin{enumerate}[start=x,label={Q\arabic*)}] for example
\usepackage{enumitem}

\usepackage{xcolor}

\usepackage{nameref}

% Important options
% in envs, use options like [baseline=x] to center on row x (or special opts t/[c]/b without 'baseline')
\usepackage{nicematrix}
\NiceMatrixOptions{cell-space-limits = 1pt}

\usepackage{booktabs}

\usepackage{tikz}
\usepackage{tikz-cd}
\usepackage{tikzsymbols}

\usepackage{pdfpages}

\usepackage[most,many,breakable]{tcolorbox}

\setlength{\parindent}{1cm}

\DeclarePairedDelimiter{\abs}{\lvert}{\rvert}

%%%%%%%%%%%%%%%%%%%%%
%%% THEOREM BOXES %%%
%%%%%%%%%%%%%%%%%%%%%
\definecolor{thmbgcol}{HTML}{aec1f9}
\definecolor{thmhlcol}{HTML}{142c72}
\definecolor{corbgcol}{HTML}{b599f7}
\definecolor{corhlcol}{HTML}{2d1760}
\definecolor{propbgcol}{HTML}{c9f7aa}
\definecolor{prophlcol}{HTML}{1e631a}
\definecolor{exbgcol}{HTML}{f7c479}
\definecolor{exhlcol}{HTML}{604419}

\definecolor{qheadcol}{HTML}{182959}

\tcbuselibrary{theorems,skins,hooks}
\newtcbtheorem[number within = section]{theorem}{Theorem}{
  enhanced, breakable, colback = thmbgcol!25,
  frame hidden, boxrule = 0sp, borderline west = {2pt}{0pt}{thmhlcol},
  sharp corners, detach title, before upper = \tcbtitle\par\smallskip,
  coltitle = thmhlcol, fonttitle = \bfseries,
  description font = \mdseries, separator sign none, segmentation style = {solid, thmhlcol}
}{th}

\newtcbtheorem[number within = section]{corollary}{Corollary}{
  enhanced, breakable, colback = corbgcol!25,
  frame hidden, boxrule = 0sp, borderline west = {2pt}{0pt}{corhlcol},
  sharp corners, detach title, before upper = \tcbtitle\par\smallskip,
  coltitle = corhlcol, fonttitle = \bfseries,
  description font = \mdseries, separator sign none, segmentation style = {solid, corhlcol}
}{cor}

\newtcbtheorem[number within = section]{proposition}{Proposition}{
  enhanced, breakable, colback = propbgcol!25,
  frame hidden, boxrule = 0sp, borderline west = {2pt}{0pt}{prophlcol},
  sharp corners, detach title, before upper = \tcbtitle\par\smallskip,
  coltitle = prophlcol, fonttitle = \bfseries,
  description font = \mdseries, separator sign none, segmentation style = {solid, prophlcol}
}{prop}

\newtcbtheorem[number within = section]{example}{Example}{
  enhanced, breakable, colback = exbgcol!25,
  frame hidden, boxrule = 0sp, borderline west = {2pt}{0pt}{exhlcol},
  sharp corners, detach title, before upper = \tcbtitle\par\smallskip,
  coltitle = exhlcol, fonttitle = \bfseries,
  description font = \mdseries, separator sign none, segmentation style = {solid, exhlcol}
}{prop}

\newtcbtheorem[number within = section]{definition}{Definition}{
  enhanced, breakable, colback = red!10,
  frame hidden, boxrule = 0sp, borderline west = {2pt}{0pt}{red!50!black},
  sharp corners, detach title, before upper = \tcbtitle\par\smallskip,
  coltitle = red!50!black, fonttitle = \bfseries,
  description font = \mdseries, separator sign none, segmentation style = {solid, exhlcol}
}{def}

\makeatletter
\newtcbtheorem{question}{Question}{enhanced,
    breakable,
    colback=white,
    colframe=qheadcol,
    attach boxed title to top left={yshift*=-\tcboxedtitleheight},
    fonttitle=\bfseries,
    title={#2},
    boxed title size=title,
    boxed title style={%
            sharp corners,
            rounded corners=northwest,
            colback=qheadcol,
            boxrule=0pt,
        },
    underlay boxed title={%
            \path[fill=tcbcolframe] (title.south west)--(title.south east)
            to[out=0, in=180] ([xshift=5mm]title.east)--
            (title.center-|frame.east)
            [rounded corners=\kvtcb@arc] |-
            (frame.north) -| cycle;
        },
    #1
}{def}
\makeatother


\begin{document}

\maketitle

\begin{theorem}{Heine-Borel (General Metric Space)}{}
  If $(X, d)$ is a compact metric space then $X$ is closed and bounded.
\end{theorem}

\begin{proof}
  Let $X$ be compact, and fix $a \in X$. Then, cover with
  \[ B_{k} = \set{ x \in X \mid d(a, x) < k } \]
  for all $k \in \NN$.
  This open cover can then be reduced to a finite cover, with $m$ being the largest-indexed open set.
  This $B_{m}$ contains $B_{n}$ for $n < m$ and so must contain all of $X$. So, $X$ is bounded.

  Suppose now that $X$ is not closed, and so there is some $y \notin X$ that is a limit point.
  Take the open cover:
  \[ C_{r} = \set{ x \in X \mid d(x, y) > r } \]
  for all $r \in \RR$. Notice that each of these sets are the complements of closed balls, and so are open themselves.
  Choose finitely many to cover $X$. Then we may pick the smallest index again, say $\delta$.
  So, if we have a sequence $\set{x_{m}} \in X$ so that $x_{m} \to y$, not all $x_{i}$ are contained in our finite cover.
  This is because for $\eps < \delta$, we cannot find any $x \in \cup C_{i}$ so that $d(x, y) < \eps$ by definition.
  So, $X$ must be closed.
\end{proof}

\begin{lemma}{}{}
  If $A \subset \RR[n]$ is compact, then $B \subset A$ is compact if $B$ is closed.
\end{lemma}
\begin{proof}
  Let the collection $C$ cover $B$. Then, the cover $C \cup \set{ \RR[n] \setminus B }$ is a cover of $A$.
  Take finitely many of these since $A$ is compact, and call this cover $C'$. This $C'$ must cover $B$ as well, since it covers $A$.
  Whether or not $C'$ contains $\RR[n] \setminus B$, the whole of $B$ must be covered by finitely many open sets still.
\end{proof}

\begin{lemma}{}{}
  For $a > 0$, $[-a, a]^{n} \subset \RR[n]$ is compact.
\end{lemma}
\begin{proof}
  Let $S_{0} = [-a, a]^{n}$ and let $C$ be an open cover. If $C$ is finite, then we are done, so assume it is infinite.
  $S_{0}$ may be subdivided into $2^{n}$ subrectangles, at least one of which needs infinitely many elements of $C$ to be covered.
  Call this subrectangle $S_{1}$. We may continue this process ad infinitum, and while we subdivide let $x_{n} \in S_{n}$ to create a sequence of points in $S_{0}$.
  We notice that this sequence must be Cauchy, as for any $n \in \NN$ this sequence eventually is wholly contained within $S_{n}$.
  Further, each has the property that
  \[ \operatorname{diam}(S_{n}) = \frac{a\sqrt{n}}{2^{n}} \]
  which can be made arbitrarily small. Therefore, $\set{x_{n}} \to x \in S_{0}$.
  This $x$ must necessarily be covered by some $U \in C$. This $U$ will also contain some ball $B$ that contains $S_{m}$ for sufficiently large $m$.
  So, we may replace the infinitely many sets covering $S_{m}$ with just $U$. So, $S_{0}$ may be covered with finitely many sets and so is compact.
\end{proof}


\begin{theorem}{Heine-Borel (Stricter)}{}
  $A \subset \RR[n]$ is compact iff it is closed and bounded.
\end{theorem}
\begin{proof}
  Let $A$ be closed and bounded. Since it is bounded, it can be enclosed by a rectangle of some size $k$:
  \[ A \subset [-k, k]^{n}. \]
  By the previous lemmas, $A$ is compact. The other direction is proven before.
\end{proof}

\end{document}
