% -*- compile-command: "latexmk -pdf document.tex" -*-
\documentclass{article}

\author{Sam Price}
\title{Reflection 13}


\newif\ifprinted%
%\printedtrue%
% Most taken from: https://github.com/SeniorMars/dotfiles/blob/main/latex_template/preamble.tex
\usepackage{geometry}

\usepackage[english]{babel}
\usepackage[T1]{fontenc}
\usepackage[utf8]{inputenc}
\usepackage{palatino}

% Use the command \doublespacing if needed
\usepackage{setspace}

\geometry{a4paper, margin=1in}

\usepackage{mathtools}
\usepackage{amssymb,amsfonts,amsthm,amsmath}

\usepackage{xfrac}
\usepackage[makeroom]{cancel}

% Use \begin{enumerate}[start=x,label={Q\arabic*)}] for example
\usepackage{enumitem}

\usepackage{xcolor}

\usepackage{nameref}

% Important options
% in envs, use options like [baseline=x] to center on row x (or special opts t/[c]/b without 'baseline')
\usepackage{nicematrix}
\NiceMatrixOptions{cell-space-limits = 1pt}

\usepackage{booktabs}

\usepackage{tikz}
\usepackage{tikz-cd}
\usepackage{tikzsymbols}

\usepackage{pdfpages}

\usepackage[most,many,breakable]{tcolorbox}

\setlength{\parindent}{1cm}

\DeclarePairedDelimiter{\abs}{\lvert}{\rvert}

%%%%%%%%%%%%%%%%%%%%%
%%% THEOREM BOXES %%%
%%%%%%%%%%%%%%%%%%%%%
\definecolor{thmbgcol}{HTML}{aec1f9}
\definecolor{thmhlcol}{HTML}{142c72}
\definecolor{corbgcol}{HTML}{b599f7}
\definecolor{corhlcol}{HTML}{2d1760}
\definecolor{propbgcol}{HTML}{c9f7aa}
\definecolor{prophlcol}{HTML}{1e631a}
\definecolor{exbgcol}{HTML}{f7c479}
\definecolor{exhlcol}{HTML}{604419}

\definecolor{qheadcol}{HTML}{182959}

\tcbuselibrary{theorems,skins,hooks}
\newtcbtheorem[number within = section]{theorem}{Theorem}{
  enhanced, breakable, colback = thmbgcol!25,
  frame hidden, boxrule = 0sp, borderline west = {2pt}{0pt}{thmhlcol},
  sharp corners, detach title, before upper = \tcbtitle\par\smallskip,
  coltitle = thmhlcol, fonttitle = \bfseries,
  description font = \mdseries, separator sign none, segmentation style = {solid, thmhlcol}
}{th}

\newtcbtheorem[number within = section]{corollary}{Corollary}{
  enhanced, breakable, colback = corbgcol!25,
  frame hidden, boxrule = 0sp, borderline west = {2pt}{0pt}{corhlcol},
  sharp corners, detach title, before upper = \tcbtitle\par\smallskip,
  coltitle = corhlcol, fonttitle = \bfseries,
  description font = \mdseries, separator sign none, segmentation style = {solid, corhlcol}
}{cor}

\newtcbtheorem[number within = section]{proposition}{Proposition}{
  enhanced, breakable, colback = propbgcol!25,
  frame hidden, boxrule = 0sp, borderline west = {2pt}{0pt}{prophlcol},
  sharp corners, detach title, before upper = \tcbtitle\par\smallskip,
  coltitle = prophlcol, fonttitle = \bfseries,
  description font = \mdseries, separator sign none, segmentation style = {solid, prophlcol}
}{prop}

\newtcbtheorem[number within = section]{example}{Example}{
  enhanced, breakable, colback = exbgcol!25,
  frame hidden, boxrule = 0sp, borderline west = {2pt}{0pt}{exhlcol},
  sharp corners, detach title, before upper = \tcbtitle\par\smallskip,
  coltitle = exhlcol, fonttitle = \bfseries,
  description font = \mdseries, separator sign none, segmentation style = {solid, exhlcol}
}{prop}

\newtcbtheorem[number within = section]{definition}{Definition}{
  enhanced, breakable, colback = red!10,
  frame hidden, boxrule = 0sp, borderline west = {2pt}{0pt}{red!50!black},
  sharp corners, detach title, before upper = \tcbtitle\par\smallskip,
  coltitle = red!50!black, fonttitle = \bfseries,
  description font = \mdseries, separator sign none, segmentation style = {solid, exhlcol}
}{def}

\makeatletter
\newtcbtheorem{question}{Question}{enhanced,
    breakable,
    colback=white,
    colframe=qheadcol,
    attach boxed title to top left={yshift*=-\tcboxedtitleheight},
    fonttitle=\bfseries,
    title={#2},
    boxed title size=title,
    boxed title style={%
            sharp corners,
            rounded corners=northwest,
            colback=qheadcol,
            boxrule=0pt,
        },
    underlay boxed title={%
            \path[fill=tcbcolframe] (title.south west)--(title.south east)
            to[out=0, in=180] ([xshift=5mm]title.east)--
            (title.center-|frame.east)
            [rounded corners=\kvtcb@arc] |-
            (frame.north) -| cycle;
        },
    #1
}{def}
\makeatother


\begin{document}

\maketitle

This week was very busy and productive, as expected for any ``break'' I get from this godforesaken place.
I worked on quite a few problems and (mostly) solved all of them:
\begin{itemize}
  \item Heine-Borel. This was exactly the boring parts I expected when going to prove this.
        The closed subset of a compact set being compact was the route I went with since it seems the most standard.
        For the forward direction, I was quite happy to use
        \[ O_{k} = \set{ x \in X \mid d(x, y) > k } \]
        since I don't think I've seen these used as the open-covering sets before.
        Not a huge detail of course, but it felt clever to me.
  \item Pullback properties on differential forms.
        This was quite simple except for the part where I brushed over something because I couldn't understand the tensor theory described earlier in the textbook.
        Is there any way to simplify this for just $f dx_{i}$ and $g dx_{j}$? Their wedge is just $fg dx_{i} \wedge dx_{j}$, but how does this split apart again?
        Maybe it's not worth overthinking and just moving on.
  \item Specific differential forms being analyzed (missing negative sign exercise).
        This is mildly embarrassing in hindsight; on PDF page 270, real page 257 in Munkres it clearly shows $d\omega$ having the old $dx_{I}$ on the outside.
        In the end, it works out because we both found my mistake and besides that slip-up the rest was perfectly fine.
  \item Arctangent differential form exercise. I like how I got to rely on the midterm quite a bit for this - certain parts were also analagous which I didn't fully get when
        doing it, which is neat. I feel like my argument became very clumsy at parts, but was correct enough that the end product was coherent.
        Mainly, going from $f|_{B}$ back to $f$ on $A$, and arguing that it cannot be continuous, and so $f$ is not differentiable i.e. $f \notin \Omega^{0}(A)$.

        For your clever shortcut, I think I'll have to look again
        at integrating exact forms over closed loops in $\RR[2]$ and whatever
        the equivalent may be in other spaces \textendash{}
        vaguely reminded me of complex integration
        from some videos I watched a while back from zetamaths,
        so it seems interesting.
        I should have slowed you down to clarify this, but we were short
        on time since I had some volume to handle.

  \item Not Very Large Commutative Diagram (NVLCD, for short).
        This one has been the ``best'' exercise I've done all semester. We'll see if raw calculations and whatever other ones I dredge out of my notes are cool.
        On the whole, I felt a toned-down version of ``elucidating much of vector calculus'' about it. I don't know if it's \emph{that} englightening, but it certainly
        felt very concrete and yet abstract enough to itch my brain the right way.
        Showing $\alpha, \beta$ were isomorphisms is something I was forgetting, but it worked out in the end.
\end{itemize}

I was also bored in WOTA after class, and worked out the $\sin(nx)\sin(mx)$ Kronecker problem.
Not super difficult, just a lookup table for trig identities since I \emph{have never} remembered those, and I don't plan on starting any time soon.
I'll have that written up for Thursday, or possibly tomorrow if I'm very ambitious.

My plans for now are the raw tangent space calculations and a couple other exercises I've found in my notes/files on my computer.

Here's to a strong end to the semester!

Album recommendation: SMILE!\ $:$D by Porter Robinson.
Probably not your style, whatever that may be, but I found it last week and I'd say it's in my top 3 of all time,
next to The Original Motion Picture Soundtrack and Under The Shade of Green.


\end{document}
