% -*- compile-command: "latexmk -pdf document.tex" -*-
\documentclass[12pt]{article}

\author{Sam Price}
\title{Only Reflection Seven\\\Large{Not Reflection Eight}}

\newif\ifprinted%
%\printedtrue%
% Most taken from: https://github.com/SeniorMars/dotfiles/blob/main/latex_template/preamble.tex
\usepackage{geometry}

\usepackage[english]{babel}
\usepackage[T1]{fontenc}
\usepackage[utf8]{inputenc}
\usepackage{palatino}

% Use the command \doublespacing if needed
\usepackage{setspace}

\geometry{a4paper, margin=1in}

\usepackage{mathtools}
\usepackage{amssymb,amsfonts,amsthm,amsmath}

\usepackage{xfrac}
\usepackage[makeroom]{cancel}

% Use \begin{enumerate}[start=x,label={Q\arabic*)}] for example
\usepackage{enumitem}

\usepackage{xcolor}

\usepackage{nameref}

% Important options
% in envs, use options like [baseline=x] to center on row x (or special opts t/[c]/b without 'baseline')
\usepackage{nicematrix}
\NiceMatrixOptions{cell-space-limits = 1pt}

\usepackage{booktabs}

\usepackage{tikz}
\usepackage{tikz-cd}
\usepackage{tikzsymbols}

\usepackage{pdfpages}

\usepackage[most,many,breakable]{tcolorbox}

\setlength{\parindent}{1cm}

\DeclarePairedDelimiter{\abs}{\lvert}{\rvert}

%%%%%%%%%%%%%%%%%%%%%
%%% THEOREM BOXES %%%
%%%%%%%%%%%%%%%%%%%%%
\definecolor{thmbgcol}{HTML}{aec1f9}
\definecolor{thmhlcol}{HTML}{142c72}
\definecolor{corbgcol}{HTML}{b599f7}
\definecolor{corhlcol}{HTML}{2d1760}
\definecolor{propbgcol}{HTML}{c9f7aa}
\definecolor{prophlcol}{HTML}{1e631a}
\definecolor{exbgcol}{HTML}{f7c479}
\definecolor{exhlcol}{HTML}{604419}

\definecolor{qheadcol}{HTML}{182959}

\tcbuselibrary{theorems,skins,hooks}
\newtcbtheorem[number within = section]{theorem}{Theorem}{
  enhanced, breakable, colback = thmbgcol!25,
  frame hidden, boxrule = 0sp, borderline west = {2pt}{0pt}{thmhlcol},
  sharp corners, detach title, before upper = \tcbtitle\par\smallskip,
  coltitle = thmhlcol, fonttitle = \bfseries,
  description font = \mdseries, separator sign none, segmentation style = {solid, thmhlcol}
}{th}

\newtcbtheorem[number within = section]{corollary}{Corollary}{
  enhanced, breakable, colback = corbgcol!25,
  frame hidden, boxrule = 0sp, borderline west = {2pt}{0pt}{corhlcol},
  sharp corners, detach title, before upper = \tcbtitle\par\smallskip,
  coltitle = corhlcol, fonttitle = \bfseries,
  description font = \mdseries, separator sign none, segmentation style = {solid, corhlcol}
}{cor}

\newtcbtheorem[number within = section]{proposition}{Proposition}{
  enhanced, breakable, colback = propbgcol!25,
  frame hidden, boxrule = 0sp, borderline west = {2pt}{0pt}{prophlcol},
  sharp corners, detach title, before upper = \tcbtitle\par\smallskip,
  coltitle = prophlcol, fonttitle = \bfseries,
  description font = \mdseries, separator sign none, segmentation style = {solid, prophlcol}
}{prop}

\newtcbtheorem[number within = section]{example}{Example}{
  enhanced, breakable, colback = exbgcol!25,
  frame hidden, boxrule = 0sp, borderline west = {2pt}{0pt}{exhlcol},
  sharp corners, detach title, before upper = \tcbtitle\par\smallskip,
  coltitle = exhlcol, fonttitle = \bfseries,
  description font = \mdseries, separator sign none, segmentation style = {solid, exhlcol}
}{prop}

\newtcbtheorem[number within = section]{definition}{Definition}{
  enhanced, breakable, colback = red!10,
  frame hidden, boxrule = 0sp, borderline west = {2pt}{0pt}{red!50!black},
  sharp corners, detach title, before upper = \tcbtitle\par\smallskip,
  coltitle = red!50!black, fonttitle = \bfseries,
  description font = \mdseries, separator sign none, segmentation style = {solid, exhlcol}
}{def}

\makeatletter
\newtcbtheorem{question}{Question}{enhanced,
    breakable,
    colback=white,
    colframe=qheadcol,
    attach boxed title to top left={yshift*=-\tcboxedtitleheight},
    fonttitle=\bfseries,
    title={#2},
    boxed title size=title,
    boxed title style={%
            sharp corners,
            rounded corners=northwest,
            colback=qheadcol,
            boxrule=0pt,
        },
    underlay boxed title={%
            \path[fill=tcbcolframe] (title.south west)--(title.south east)
            to[out=0, in=180] ([xshift=5mm]title.east)--
            (title.center-|frame.east)
            [rounded corners=\kvtcb@arc] |-
            (frame.north) -| cycle;
        },
    #1
}{def}
\makeatother


\begin{document}

\maketitle

This week was rough in terms of math.
Everything was due today across so many classes, so let's start with the midterm.

I spent a lot of time on the midterm; it ate up the majority of my days Friday to Sunday,
usually from 10 to 5 or so, depending on what I had going on.
As I do more and more linear algebra it becomes obvious to me that
(a) I really need to take the class with you next semester, and
(b) that it \emph{always} comes down to counting elements in a basis
and doing something clever with them. It makes sense, because that is quite literally the basis,
pun intended, of linear algebra for the most part.
When looking at orthogonal complement stuff, it in fact was Gilbert Strang's
little four-subspaces diatribe that I ended up reading, so the picture you drew was definitely
the one that was familiar to me. I agree though in that the picture doesn't really elucidate
anything more than the words already do.

Besides the first problem and my struggle to generalize at the end of the second
(which I want to see more explicitly when we go over the midterm)
the midterm was just a lot of thinking rather than trying to write down things explicitly.
The third problem felt the best, although as I'm sure you'll notice I get a little hand-wavy
in that I mostly just\ldots ignore the differential forms
and fall back on our current integral notation.
If it's proper and we can treat those differential forms as commutative with the functions,
then it's all fine and nothing really goes wrong.

Iain and I did work on a couple problems.
As you mentioned, $\abs{x}$ integrated twice would be a sufficient $C^{2}$
but not $C^{3}$ function.
I wrote it up but since you gave a solution I won't try presenting it.
We also figured out the cross not being homeomorphic to a line:
I'll present this later but it boiled down to when the origin is removed,
there are four disconnected components of the axes and only two on $\RR$.
Since the number of connected components needs to be preserved under a homeomorphism,
these curves are not homeomorphic. Four explicit disjoint open sets are given to show why
the cross sans origin is 4 components. It took a minute to realize it but the
right triangles that are bisected by the axes suffice.

In terms of capstone work, I need to get on it.
I am heavily debating getting a copy of Tao's Analysis II book or borrowing
it from the UW Madison system library (can we get it delivered to UW-Whitewater?)
with the only upside to buying it being adding another fun book to my collection.
Otherwise, Anna's Archive does exist.
I have gotten a fleshed-out proof of the relevant fixed-point theorem
and so feel confident in actually doing it.

Difficulties this week are getting cleared up as we go.
I think the tricky part thus far is mostly (yet again) the formalization process
of saying that a manifold ``looks like'' $\RR[n]$ locally.
It makes sense that every point has to have a open set around it that is
homeomorphic to an open set in the relevant Euclidean space, so I can justify the
intuition that way.
The less explicit (implicit, even?) definitions still seem a bit vague but I guess we'll
see the equivalence of the parametric and implicit manifold definitions Thursday.

The other big hangup I have at this point in the semester is still derivative related,
which is a little sad but it's still unintuitive.
Iain and I discussed this a little but iterated derivatives feel very intimidating since
they grow in dimension (they become tensors, right?)
Is there anything useful to be gleaned from them once we use them to show Fubini?

Towards the end of our meeting today you said ``It's about What You Know''.
Any indiscriminantly opinionated hot takes you'd like to share while on the job?
Not politically, of course, but about Two Door Cinema Club?
%Currently really into experimental electronic music but TDCC is fantastic.
%``eyes in the water'', ``!!`! find the answer !`!!`'', ``EPSILON LOVE'' and ``Toy Car''
%are my favorites by far right now.

Similarly unrelated, does it feel like I overuse parentheses a lot while writing?
It's something I've been trying to avoid recently and I am not sure how much it is working.


\end{document}
