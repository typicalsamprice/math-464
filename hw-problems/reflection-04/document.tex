% -*- compile-command: "latexmk -pdf document.tex" -*-
\documentclass[12pt]{article}

/home/sam/Git/latex-template/template.tex
\togglefalse{paper}

%%%%%%%%%%%%%%%%%%%%%
%%% THEOREM BOXES %%%
%%%%%%%%%%%%%%%%%%%%%
\usepackage[most,many,breakable]{tcolorbox}
\tcbuselibrary{theorems,skins,hooks}

% Stolen from: https://tex.stackexchange.com/a/330460
\makeatletter
\renewenvironment{proof}[1][\proofname]{\par
  \pushQED{\qed}%
  \normalfont \topsep6\p@\@plus6\p@\relax
  \trivlist
  \item[%
    \hskip\labelsep
    \normalfont\bfseries % was \itshape
    #1%
    \@addpunct{.}% remove this if you don't want punctuation
  ]\ignorespaces
}{%
  \popQED\endtrivlist\@endpefalse
}
\let\qed\relax % avoid a warning
\DeclareRobustCommand{\qed}{%
  \ifmmode \mathqed
  \else
    \leavevmode\unskip\penalty\@M\hbox{}\nobreak\hspace{.5em minus .1em}% was \hfill
    \hbox{\qedsymbol}%
  \fi
}
\makeatother

\ifprinted%
  \colorlet{thmbgcol}{white}
  \colorlet{lembgcol}{white}
  \colorlet{corbgcol}{white}
  \colorlet{propbgcol}{white}
  \colorlet{exbgcol}{white}
  \colorlet{defbgcol}{white}

  \colorlet{qheadcol}{black!20!white}
  \colorlet{qheadtxtcol}{black!90}

  \colorlet{exhlcol}{black}
  \colorlet{prophlcol}{black}
  \colorlet{corhlcol}{black}
  \colorlet{thmhlcol}{black}
  \colorlet{lemhlcol}{black}
  \colorlet{defhlcol}{black}
\else
  \definecolor{thmbgcol}{HTML}{aec1f9}
  \definecolor{corbgcol}{HTML}{b599f7}
  \definecolor{propbgcol}{HTML}{c9f7aa}
  \definecolor{exbgcol}{HTML}{f7c479}
  \colorlet{defbgcol}{red!7}

  \definecolor{qheadcol}{HTML}{182959}
  \colorlet{qheadtxtcol}{white}

  \definecolor{exhlcol}{HTML}{604419}
  \definecolor{prophlcol}{HTML}{1e631a}
  \definecolor{corhlcol}{HTML}{2d1760}
  \definecolor{thmhlcol}{HTML}{142c72}
  \colorlet{defhlcol}{red!50!black}

  \definecolor{lembgcol}{HTML}{e998f2}
  \definecolor{lemhlcol}{HTML}{791684}
\fi

\newtcbtheorem[number within = section]{theorem}{Theorem}{
  enhanced, breakable, colback = thmbgcol!20,
  frame hidden, boxrule = 0sp, borderline west = {2pt}{0pt}{thmhlcol},
  sharp corners, detach title, before upper = \tcbtitle\par\smallskip,
  coltitle = thmhlcol, fonttitle = \bfseries,
  description font = \mdseries, separator sign none, segmentation style = {solid, thmhlcol}
}{th}

\newtcbtheorem[number within = section]{lemma}{Lemma}{
  enhanced, breakable, colback = lembgcol!20,
  frame hidden, boxrule = 0sp, borderline west = {2pt}{0pt}{lemhlcol},
  sharp corners, detach title, before upper = \tcbtitle\par\smallskip,
  coltitle = lemhlcol, fonttitle = \bfseries,
  description font = \mdseries, separator sign none, segmentation style = {solid, lemhlcol}
}{lem}

\newtcbtheorem[number within = section]{corollary}{Corollary}{
  enhanced, breakable, colback = corbgcol!20,
  frame hidden, boxrule = 0sp, borderline west = {2pt}{0pt}{corhlcol},
  sharp corners, detach title, before upper = \tcbtitle\par\smallskip,
  coltitle = corhlcol, fonttitle = \bfseries,
  description font = \mdseries, separator sign none, segmentation style = {solid, corhlcol}
}{cor}

\newtcbtheorem[number within = section]{proposition}{Proposition}{
  enhanced, breakable, colback = propbgcol!25,
  frame hidden, boxrule = 0sp, borderline west = {2pt}{0pt}{prophlcol},
  sharp corners, detach title, before upper = \tcbtitle\par\smallskip,
  coltitle = prophlcol, fonttitle = \bfseries,
  description font = \mdseries, separator sign none, segmentation style = {solid, prophlcol}
}{prop}

\newtcbtheorem[number within = section]{example}{Example}{
  enhanced, breakable, colback = exbgcol!20,
  frame hidden, boxrule = 0sp, borderline west = {2pt}{0pt}{exhlcol},
  sharp corners, detach title, before upper = \tcbtitle\par\smallskip,
  coltitle = exhlcol, fonttitle = \bfseries,
  description font = \mdseries, separator sign none, segmentation style = {solid, exhlcol}
}{prop}

\newtcbtheorem[number within = section]{definition}{Definition}{
  enhanced, breakable, colback = defbgcol,
  frame hidden, boxrule = 0sp, borderline west = {2pt}{0pt}{defhlcol},
  sharp corners, detach title, before upper = \tcbtitle\par\smallskip,
  coltitle = defhlcol, fonttitle = \bfseries,
  description font = \mdseries, separator sign none, segmentation style = {solid, exhlcol}
}{def}

\makeatletter
\newtcbtheorem{question}{Question}{enhanced,
    breakable,
    colback=white,
    colframe=qheadcol,
    attach boxed title to top left={yshift*=-\tcboxedtitleheight},
    fonttitle=\bfseries,
    coltitle=qheadtxtcol,
    title={#2},
    boxed title size=title,
    boxed title style={%
            sharp corners,
            rounded corners=northwest,
            colback=qheadcol,
            boxrule=0pt,
        },
    underlay boxed title={%
            \path[fill=tcbcolframe] (title.south west)--(title.south east)
            to[out=0, in=180] ([xshift=5mm]title.east)--
            (title.center-|frame.east)
            [rounded corners=\kvtcb@arc] |-
            (frame.north) -| cycle;
        },
    #1
}{}
\makeatother


\author{Sam Price}
\date{}
\title{math four sixty-four, reflection four}

\begin{document}

\maketitle

This week has been semi-productive in terms of math. Naturally instead of being \emph{just} busy, I was/am also slightly sick and so am worrying simultaneously about that.

In terms of 281 (general problem solving) I've gotten somewhat comfortable with a couple of the polynomial problems of this week, and have come up with a proof of uniqueness for something
which is quite exciting since I don't get the chance to do that often. In other non-464 classes nothing particularly special has happened.

In 464 I've gotten much more comfortable with integration in $\RR[n]$,
since as you said it is much nicer than differentiation.
I got back into the swing of things dealing with partitions and their subrectangles,
and all of the arguments using refinements or general partitions $P_{1},P_{2}$ to argue
for integral inequalities made sense without looking at them too hard at all.
I also appreciate the fact we are ``using'' measures at all, since even without anything rigorous
(Tao's measure theory book scares me) it seems intuitive and obvious that something ``infinitely thin''
contributes no volume to a total integral. Proving generally easy ideas and then turning and saying
that $\int\limits_{S}\! f = \int\limits_{\Int S}\!\!\! f$ \emph{of course} is really satisfying to me.

The difficulties this week come in the form of not understanding why we are extending Fubini to
simple regions. I get that these are regions that are ``nice'' because we can basically do Calc III
integration to them, but is that really it? I get that they are \emph{simple} for a reason however it
feels lackluster for this to be their only real use.
Further, looking at this would it be useful to just leverage the definition to show that $\RR[n - 1] \times \set{0}$
is rectifiable in $\RR[n]$ since you could use $\phi = \psi = 0$ as constant functions?

As a question, is a compact set necessarily rectifiable in $\RR[n]$?
Given the Koch snowflake example I imagine not but off the top of my head I can't imagine why.

My work thus far on my inverse function theorem proof is still fairly slight,
although I have the Banach fixed-point theorem down and understood,
the proof(s) by successive approximation also look interesting.
Munkres' approach (page 65, for my own future reference)
with many parts might be more fun simply because I have 45 minutes on final day that I plan to \emph{use}.

Reading ahead into Munkres, I can't wait for (dreading) the change of variables we will do.
Improper integrals seems strange enough, but even in Calc III I had issues with keeping track of all this stuff
just between polar and cylindrical. Because this will be substantially more general (I imagine)
it might be simpler just from that perspective.

A reason given by AI that the Koch snowflake isn't rectifiable is that its perimeter grows by a factor of $4/3$
at each iteration, and so has infinite perimeter.
That doesn't feel satisfactory though, and I bet there's a way to show it isn't rectifiable
without directly showing that every countable cover of $K$ is ``big''.

\end{document}
