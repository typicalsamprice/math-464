% -*- compile-command: "latexmk -pdf document.tex" -*-
\documentclass{article}

\author{Sam Price}
\title{Derivative Shenanigans}

\newif\ifprinted%
\printedtrue%
% Most taken from: https://github.com/SeniorMars/dotfiles/blob/main/latex_template/preamble.tex
\usepackage{geometry}

\usepackage[english]{babel}
\usepackage[T1]{fontenc}
\usepackage[utf8]{inputenc}
\usepackage{palatino}

% Use the command \doublespacing if needed
\usepackage{setspace}

\geometry{a4paper, margin=1in}

\usepackage{mathtools}
\usepackage{amssymb,amsfonts,amsthm,amsmath}

\usepackage{xfrac}
\usepackage[makeroom]{cancel}

% Use \begin{enumerate}[start=x,label={Q\arabic*)}] for example
\usepackage{enumitem}

\usepackage{xcolor}

\usepackage{nameref}

% Important options
% in envs, use options like [baseline=x] to center on row x (or special opts t/[c]/b without 'baseline')
\usepackage{nicematrix}
\NiceMatrixOptions{cell-space-limits = 1pt}

\usepackage{booktabs}

\usepackage{tikz}
\usepackage{tikz-cd}
\usepackage{tikzsymbols}

\usepackage{pdfpages}

\usepackage[most,many,breakable]{tcolorbox}

\setlength{\parindent}{1cm}

\DeclarePairedDelimiter{\abs}{\lvert}{\rvert}

%%%%%%%%%%%%%%%%%%%%%
%%% THEOREM BOXES %%%
%%%%%%%%%%%%%%%%%%%%%
\definecolor{thmbgcol}{HTML}{aec1f9}
\definecolor{thmhlcol}{HTML}{142c72}
\definecolor{corbgcol}{HTML}{b599f7}
\definecolor{corhlcol}{HTML}{2d1760}
\definecolor{propbgcol}{HTML}{c9f7aa}
\definecolor{prophlcol}{HTML}{1e631a}
\definecolor{exbgcol}{HTML}{f7c479}
\definecolor{exhlcol}{HTML}{604419}

\definecolor{qheadcol}{HTML}{182959}

\tcbuselibrary{theorems,skins,hooks}
\newtcbtheorem[number within = section]{theorem}{Theorem}{
  enhanced, breakable, colback = thmbgcol!25,
  frame hidden, boxrule = 0sp, borderline west = {2pt}{0pt}{thmhlcol},
  sharp corners, detach title, before upper = \tcbtitle\par\smallskip,
  coltitle = thmhlcol, fonttitle = \bfseries,
  description font = \mdseries, separator sign none, segmentation style = {solid, thmhlcol}
}{th}

\newtcbtheorem[number within = section]{corollary}{Corollary}{
  enhanced, breakable, colback = corbgcol!25,
  frame hidden, boxrule = 0sp, borderline west = {2pt}{0pt}{corhlcol},
  sharp corners, detach title, before upper = \tcbtitle\par\smallskip,
  coltitle = corhlcol, fonttitle = \bfseries,
  description font = \mdseries, separator sign none, segmentation style = {solid, corhlcol}
}{cor}

\newtcbtheorem[number within = section]{proposition}{Proposition}{
  enhanced, breakable, colback = propbgcol!25,
  frame hidden, boxrule = 0sp, borderline west = {2pt}{0pt}{prophlcol},
  sharp corners, detach title, before upper = \tcbtitle\par\smallskip,
  coltitle = prophlcol, fonttitle = \bfseries,
  description font = \mdseries, separator sign none, segmentation style = {solid, prophlcol}
}{prop}

\newtcbtheorem[number within = section]{example}{Example}{
  enhanced, breakable, colback = exbgcol!25,
  frame hidden, boxrule = 0sp, borderline west = {2pt}{0pt}{exhlcol},
  sharp corners, detach title, before upper = \tcbtitle\par\smallskip,
  coltitle = exhlcol, fonttitle = \bfseries,
  description font = \mdseries, separator sign none, segmentation style = {solid, exhlcol}
}{prop}

\newtcbtheorem[number within = section]{definition}{Definition}{
  enhanced, breakable, colback = red!10,
  frame hidden, boxrule = 0sp, borderline west = {2pt}{0pt}{red!50!black},
  sharp corners, detach title, before upper = \tcbtitle\par\smallskip,
  coltitle = red!50!black, fonttitle = \bfseries,
  description font = \mdseries, separator sign none, segmentation style = {solid, exhlcol}
}{def}

\makeatletter
\newtcbtheorem{question}{Question}{enhanced,
    breakable,
    colback=white,
    colframe=qheadcol,
    attach boxed title to top left={yshift*=-\tcboxedtitleheight},
    fonttitle=\bfseries,
    title={#2},
    boxed title size=title,
    boxed title style={%
            sharp corners,
            rounded corners=northwest,
            colback=qheadcol,
            boxrule=0pt,
        },
    underlay boxed title={%
            \path[fill=tcbcolframe] (title.south west)--(title.south east)
            to[out=0, in=180] ([xshift=5mm]title.east)--
            (title.center-|frame.east)
            [rounded corners=\kvtcb@arc] |-
            (frame.north) -| cycle;
        },
    #1
}{def}
\makeatother


\begin{document}

\maketitle

\begin{proposition}{}{}
  Let $f \from \RR[m] \to \RR[n]$ be differentiable.
  Then,
  \begin{enumerate}[start=1,label={(\alph*)}]
    \item $Df = 0$ uniformly iff $f$ is constant.
    \item For some $n \times m$ matrix $A$, $Df = A$ uniformly iff $f$ is affine. That is,
          \[ f(x) = Ax + c \]
          for some $c \in \RR[n]$.
  \end{enumerate}
\end{proposition}

\begin{lemma}{}{}
  Let $g \from [a, b] \to \RR$ be differentiable on $(a, b)$.
  If $g'(x) = 0$ for all $x \in D$, then $g$ is constant.
\end{lemma}
\begin{proof}[Proof of Lemma]
  Let $x \in [a, b]$. Of course by the mean value theorem we have some $y \in (a, x)$ so that
  \[ g'(y) = \frac{g(x) - g(a)}{x - a}. \]
  Since $g' = 0$, then $g(x) - g(a) = 0$ and thus $g(x) = g(a)$ for all $x \in (a, b)$.
  Doing the same from the other side shows that $g(a) = g(b)$, and so $g$ is constant.
\end{proof}

\begin{proof}[Proof of (a)]
  The backwards direction is true by simple calculation; any constant function has uniformly derivative 0.
  Therefore, we concern ourselves now with the forward direction.

  Let $f \from \RR[m] \to \RR[n]$ be differentiable, and $Df = 0$ uniformly.
  This requires that $Df_{i} = 0$ for all $1 \le i \le n$, and so let us show that $g = f_{i}$ is constant on $\RR[m]$.
  Suppose $g(x) \ne g(y)$ for some $x, y \in \RR[m]$, and define a sequence of points:
  \begin{align*}
    v_{0} &= (x_{1}, \ldots, x_{m}) = x\\
    v_{1} &= (y_{1}, x_{2}, \ldots, x_{m})\\
    v_{2} &= (y_{1}, y_{2}, x_{3}, \ldots, x_{m})\\
    v_{m} &= (y_{1}, \ldots, y_{m}) = y.
  \end{align*}
  Along the line from $v_{0}$ to $v_{1}$ for instance, we follow the $x_{1}$-axis, and so we can create
  (WLOG $x_{1} < y_{1}$)
  \[ h \from [x_{1}, y_{1}] \to \RR \]
  and we observe
  \[ h(x_{1}) = g(v_{0}) \qquad \text{ and } \qquad h(y_{1}) = g(v_{1}) \]
  and further that $h = g$ on $[x_{1}, y_{1}] \times (x_{2}, \ldots, x_{m})$.
  By direct calculation of course, $h' = \partial g/\partial x_{1} = 0$, and we may use our lemma to note $h(x_{1}) = h(y_{1})$, and so $g(v_{0}) = g(v_{1})$.
  For each pair of points $v_{k}, v_{k + 1}$ we can define such a transition function, and by iterating we find then that
  \[ g(v_{0}) = g(v_{1}) = \cdots = g(v_{i}) = \cdots = g(v_{m}). \]
  However, $v_{0} = x$ and $v_{m} = y$ and we reach a contradiction where $g(x) = g(y)$.
  With $g$ constant, every component of $f$ must be and thus $f$ as a whole.
\end{proof}

\begin{proof}[Proof of (b)]
  The backwards direction is again true by calculation. We know for such an $f$ that
  \[ f_{i} = \sum_{j = 1}^{m}a_{ij}x_{j} + c_{j} \]
  and so $a_{ij}$ shows up exactly for $D_{j}f_{i}$, showing $Df = A$.

  Suppose now that $Df = A$ uniformly. Let $h(x) = f(x) - Ax$, and notice that $h' = 0$ everywhere.
  By the above, $h = c$ for some $c \in \RR[n]$ and so $f(x) - Ax = c$ which means $f(x) = Ax + c$.
\end{proof}

\end{document}
