\documentclass[12pt]{article}

\author{Sam Price}
\date{}% Clear that out
\title{Math 464 Reflection 5}

% Most taken from: https://github.com/SeniorMars/dotfiles/blob/main/latex_template/preamble.tex
\usepackage{geometry}

\usepackage[english]{babel}
\usepackage[T1]{fontenc}
\usepackage[utf8]{inputenc}
\usepackage{palatino}

% Use the command \doublespacing if needed
\usepackage{setspace}

\geometry{a4paper, margin=1in}

\usepackage{mathtools}
\usepackage{amssymb,amsfonts,amsthm,amsmath}

\usepackage{xfrac}
\usepackage[makeroom]{cancel}

% Use \begin{enumerate}[start=x,label={Q\arabic*)}] for example
\usepackage{enumitem}

\usepackage{xcolor}

\usepackage{nameref}

% Important options
% in envs, use options like [baseline=x] to center on row x (or special opts t/[c]/b without 'baseline')
\usepackage{nicematrix}
\NiceMatrixOptions{cell-space-limits = 1pt}

\usepackage{booktabs}

\usepackage{tikz}
\usepackage{tikz-cd}
\usepackage{tikzsymbols}

\usepackage{pdfpages}

\usepackage[most,many,breakable]{tcolorbox}

\setlength{\parindent}{1cm}

\DeclarePairedDelimiter{\abs}{\lvert}{\rvert}

%%%%%%%%%%%%%%%%%%%%%
%%% THEOREM BOXES %%%
%%%%%%%%%%%%%%%%%%%%%
\definecolor{thmbgcol}{HTML}{aec1f9}
\definecolor{thmhlcol}{HTML}{142c72}
\definecolor{corbgcol}{HTML}{b599f7}
\definecolor{corhlcol}{HTML}{2d1760}
\definecolor{propbgcol}{HTML}{c9f7aa}
\definecolor{prophlcol}{HTML}{1e631a}
\definecolor{exbgcol}{HTML}{f7c479}
\definecolor{exhlcol}{HTML}{604419}

\definecolor{qheadcol}{HTML}{182959}

\tcbuselibrary{theorems,skins,hooks}
\newtcbtheorem[number within = section]{theorem}{Theorem}{
  enhanced, breakable, colback = thmbgcol!25,
  frame hidden, boxrule = 0sp, borderline west = {2pt}{0pt}{thmhlcol},
  sharp corners, detach title, before upper = \tcbtitle\par\smallskip,
  coltitle = thmhlcol, fonttitle = \bfseries,
  description font = \mdseries, separator sign none, segmentation style = {solid, thmhlcol}
}{th}

\newtcbtheorem[number within = section]{corollary}{Corollary}{
  enhanced, breakable, colback = corbgcol!25,
  frame hidden, boxrule = 0sp, borderline west = {2pt}{0pt}{corhlcol},
  sharp corners, detach title, before upper = \tcbtitle\par\smallskip,
  coltitle = corhlcol, fonttitle = \bfseries,
  description font = \mdseries, separator sign none, segmentation style = {solid, corhlcol}
}{cor}

\newtcbtheorem[number within = section]{proposition}{Proposition}{
  enhanced, breakable, colback = propbgcol!25,
  frame hidden, boxrule = 0sp, borderline west = {2pt}{0pt}{prophlcol},
  sharp corners, detach title, before upper = \tcbtitle\par\smallskip,
  coltitle = prophlcol, fonttitle = \bfseries,
  description font = \mdseries, separator sign none, segmentation style = {solid, prophlcol}
}{prop}

\newtcbtheorem[number within = section]{example}{Example}{
  enhanced, breakable, colback = exbgcol!25,
  frame hidden, boxrule = 0sp, borderline west = {2pt}{0pt}{exhlcol},
  sharp corners, detach title, before upper = \tcbtitle\par\smallskip,
  coltitle = exhlcol, fonttitle = \bfseries,
  description font = \mdseries, separator sign none, segmentation style = {solid, exhlcol}
}{prop}

\newtcbtheorem[number within = section]{definition}{Definition}{
  enhanced, breakable, colback = red!10,
  frame hidden, boxrule = 0sp, borderline west = {2pt}{0pt}{red!50!black},
  sharp corners, detach title, before upper = \tcbtitle\par\smallskip,
  coltitle = red!50!black, fonttitle = \bfseries,
  description font = \mdseries, separator sign none, segmentation style = {solid, exhlcol}
}{def}

\makeatletter
\newtcbtheorem{question}{Question}{enhanced,
    breakable,
    colback=white,
    colframe=qheadcol,
    attach boxed title to top left={yshift*=-\tcboxedtitleheight},
    fonttitle=\bfseries,
    title={#2},
    boxed title size=title,
    boxed title style={%
            sharp corners,
            rounded corners=northwest,
            colback=qheadcol,
            boxrule=0pt,
        },
    underlay boxed title={%
            \path[fill=tcbcolframe] (title.south west)--(title.south east)
            to[out=0, in=180] ([xshift=5mm]title.east)--
            (title.center-|frame.east)
            [rounded corners=\kvtcb@arc] |-
            (frame.north) -| cycle;
        },
    #1
}{def}
\makeatother

\begin{document}

\maketitle

This week has been fairly uninteresting in terms of math, maybe as usual.
The most influential part has been reading the textbook.
Reading past the integration section into the Change of Variables chapter,
it first talks about partitions of unity much like you mentioned today.
I find the connection between the extended integral and such partitions interesting
since initially they seem insanely different.
I can also see why the Change of Variables Theorem is left as a capstone project,
it looks very long and intimidating at a glance since it splits so many things up
and Lemma-fies things to make it (somewhat) nicer to prove.

% TODO Show that Transpose(M) = Inverse(M) for an orthogonal map M
% Note orthogonal maps are of the form Mx for some M in O(n).

The exercises on orthogonal transformations are interesting too.
They aren't particularly hard (at least the ones I did quickly, obviously)
but it is definitely interesting how much of math can be reduced in a useful way
to good 'ol matrices. I'm going to go back and do some integral exercises to understand
some of this stuff more before the midterm; it would be good for me to do some
derivative stuff as well since that's the topic I understand the least right now.
For raw calcuation \emph{of} a given function's derivative it isn't too difficult,
but getting the hang of exercises like 9.4.5 where actually showing properties of them
is necessary. That one in particular looks interesting since it is (almost) talking
about varities in $\mathbb{R}^3$, although it skips out on solely restricting
$f, g$ to polynomials in $\mathbb{R}\lbrack x, y, z\rbrack$.


I remember watching a video that proves
(in a somewhat YouTube-ified/immature way by likening contradiction to "proving the devil wrong")
that a rotation about the origin and translation composed
is simply a rotation about some other point.
The edge case would be a translation being equivalent to a "rotation about a point at infinity"
but I wonder if this can be rephrased in terms of these concepts and solved in a linear
algebra-esque way. That video also only considers the case in $\mathbb{R}^2$ since it plays
around in the plane; this could definitely be translated to higher dimensions in some fashion I'm sure.
The only issue I believe one might run into is the fact that $\operatorname{SO}(n)$
is nonabelian for $n > 2$, which are the cases we want.

Using the determinant \emph{as} the measure of volume also gives more "intuition" or at least
reasoning for why subspaces of dimension $< n$ are measure zero in $\mathbb{R}^n$.
My intuition was always "why can't we just cover something like a plane in $\mathbb{R}^3$ with squares
just like when integrating in $\mathbb{R}^2$ and use that same value?
Looking forward to chapter 5, turns out that's exactly what manifolds are for in the first place.
This makes more sense to me since it feels incredibly useless to \emph{always} shove a line and plane into the same category:
one is clearly "bigger" intuitively and so should live in some larger space
rather than just being compacted into the measure-zero behemoth.

I also have a solution to the question of if $f, g \colon \mathbb{R}^n \to \mathbb{R}$
are functions such that $f \ne g$ only on a measure-zero set $D$ and $f$ is continuous,
can $g$ be continuous? 
The answer is no, and is shown by picking $d \in D$.
Then, we can have $(x_n) \to d$ with $f(x_n) \to f(d)$.
However, we may also pick our sequence so no $x_i \in D$,
since $D$ may not contain a whole open interval it may be at "most" dense.
This means that $x_n \to d$ but $g(x_n) \not\to g(d)$ since $g(x_n) = f(x_n) \to f(d)$
and so $g$ cannot be continuous at any $d$, but is continuous everywhere else.

\end{document}
