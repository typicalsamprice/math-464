% -*- compile-command: "latexmk -pdf document.tex" -*-
\documentclass{article}

\author{Sam Price}
\title{Large Commutative Diagram}

\newif\ifprinted%
\printedtrue%
% Most taken from: https://github.com/SeniorMars/dotfiles/blob/main/latex_template/preamble.tex
\usepackage{geometry}

\usepackage[english]{babel}
\usepackage[T1]{fontenc}
\usepackage[utf8]{inputenc}
\usepackage{palatino}

% Use the command \doublespacing if needed
\usepackage{setspace}

\geometry{a4paper, margin=1in}

\usepackage{mathtools}
\usepackage{amssymb,amsfonts,amsthm,amsmath}

\usepackage{xfrac}
\usepackage[makeroom]{cancel}

% Use \begin{enumerate}[start=x,label={Q\arabic*)}] for example
\usepackage{enumitem}

\usepackage{xcolor}

\usepackage{nameref}

% Important options
% in envs, use options like [baseline=x] to center on row x (or special opts t/[c]/b without 'baseline')
\usepackage{nicematrix}
\NiceMatrixOptions{cell-space-limits = 1pt}

\usepackage{booktabs}

\usepackage{tikz}
\usepackage{tikz-cd}
\usepackage{tikzsymbols}

\usepackage{pdfpages}

\usepackage[most,many,breakable]{tcolorbox}

\setlength{\parindent}{1cm}

\DeclarePairedDelimiter{\abs}{\lvert}{\rvert}

%%%%%%%%%%%%%%%%%%%%%
%%% THEOREM BOXES %%%
%%%%%%%%%%%%%%%%%%%%%
\definecolor{thmbgcol}{HTML}{aec1f9}
\definecolor{thmhlcol}{HTML}{142c72}
\definecolor{corbgcol}{HTML}{b599f7}
\definecolor{corhlcol}{HTML}{2d1760}
\definecolor{propbgcol}{HTML}{c9f7aa}
\definecolor{prophlcol}{HTML}{1e631a}
\definecolor{exbgcol}{HTML}{f7c479}
\definecolor{exhlcol}{HTML}{604419}

\definecolor{qheadcol}{HTML}{182959}

\tcbuselibrary{theorems,skins,hooks}
\newtcbtheorem[number within = section]{theorem}{Theorem}{
  enhanced, breakable, colback = thmbgcol!25,
  frame hidden, boxrule = 0sp, borderline west = {2pt}{0pt}{thmhlcol},
  sharp corners, detach title, before upper = \tcbtitle\par\smallskip,
  coltitle = thmhlcol, fonttitle = \bfseries,
  description font = \mdseries, separator sign none, segmentation style = {solid, thmhlcol}
}{th}

\newtcbtheorem[number within = section]{corollary}{Corollary}{
  enhanced, breakable, colback = corbgcol!25,
  frame hidden, boxrule = 0sp, borderline west = {2pt}{0pt}{corhlcol},
  sharp corners, detach title, before upper = \tcbtitle\par\smallskip,
  coltitle = corhlcol, fonttitle = \bfseries,
  description font = \mdseries, separator sign none, segmentation style = {solid, corhlcol}
}{cor}

\newtcbtheorem[number within = section]{proposition}{Proposition}{
  enhanced, breakable, colback = propbgcol!25,
  frame hidden, boxrule = 0sp, borderline west = {2pt}{0pt}{prophlcol},
  sharp corners, detach title, before upper = \tcbtitle\par\smallskip,
  coltitle = prophlcol, fonttitle = \bfseries,
  description font = \mdseries, separator sign none, segmentation style = {solid, prophlcol}
}{prop}

\newtcbtheorem[number within = section]{example}{Example}{
  enhanced, breakable, colback = exbgcol!25,
  frame hidden, boxrule = 0sp, borderline west = {2pt}{0pt}{exhlcol},
  sharp corners, detach title, before upper = \tcbtitle\par\smallskip,
  coltitle = exhlcol, fonttitle = \bfseries,
  description font = \mdseries, separator sign none, segmentation style = {solid, exhlcol}
}{prop}

\newtcbtheorem[number within = section]{definition}{Definition}{
  enhanced, breakable, colback = red!10,
  frame hidden, boxrule = 0sp, borderline west = {2pt}{0pt}{red!50!black},
  sharp corners, detach title, before upper = \tcbtitle\par\smallskip,
  coltitle = red!50!black, fonttitle = \bfseries,
  description font = \mdseries, separator sign none, segmentation style = {solid, exhlcol}
}{def}

\makeatletter
\newtcbtheorem{question}{Question}{enhanced,
    breakable,
    colback=white,
    colframe=qheadcol,
    attach boxed title to top left={yshift*=-\tcboxedtitleheight},
    fonttitle=\bfseries,
    title={#2},
    boxed title size=title,
    boxed title style={%
            sharp corners,
            rounded corners=northwest,
            colback=qheadcol,
            boxrule=0pt,
        },
    underlay boxed title={%
            \path[fill=tcbcolframe] (title.south west)--(title.south east)
            to[out=0, in=180] ([xshift=5mm]title.east)--
            (title.center-|frame.east)
            [rounded corners=\kvtcb@arc] |-
            (frame.north) -| cycle;
        },
    #1
}{def}
\makeatother


\begin{document}

\maketitle

Show that
% https://q.uiver.app/#q=WzAsOCxbMCwwLCJTIl0sWzAsMiwiViJdLFsyLDAsIlxcT21lZ2FeMChBKSJdLFsyLDIsIlxcT21lZ2FeMShBKSJdLFswLDQsIlYiXSxbMCw2LCJTIl0sWzIsNiwiXFxPbWVnYV4zKEEpIl0sWzIsNCwiXFxPbWVnYV4yKEEpIl0sWzAsMV0sWzAsMl0sWzIsM10sWzEsM10sWzEsNF0sWzQsNV0sWzUsNl0sWzQsN10sWzcsNl0sWzMsN11d
\[\begin{tikzcd}
  S && {\Omega^0(A)} \\
  \\
  V && {\Omega^1(A)} \\
  \\
  V && {\Omega^2(A)} \\
  \\
  S && {\Omega^3(A)}
  \arrow["\imath", from=1-1, to=1-3]
  \arrow["\operatorname{grad}"', from=1-1, to=3-1]
  \arrow["d", from=1-3, to=3-3]
  \arrow["\imath", from=3-1, to=3-3]
  \arrow["\operatorname{curl}"', from=3-1, to=5-1]
  \arrow["d", from=3-3, to=5-3]
  \arrow["\alpha", from=5-1, to=5-3]
  \arrow["\operatorname{div}"', from=5-1, to=7-1]
  \arrow["d", from=5-3, to=7-3]
  \arrow["\beta", from=7-1, to=7-3]
\end{tikzcd}\]
commutes, where $A \subset \RR[3]$ is open, $S$ are scalar fields on $A$ and $V$ are the vector fields on $A$.

We first show the top square commutes for $A \subset \RR[n]$, then the bottom square works for $\Omega^{n - 1}(A) \to \Omega^{n}(A)$ for general $n$, and
then that the center works for the exact numbers shown.

\underline{Step 1 (Top Square)}:
Note that for this, $A \subset \RR[n]$ generally.
We may have the inclusion $S \to \Omega^{0}(A)$ by sending $f \mapsto f$, since it is a valid 0-form already.
Similarly, for some vector field $G(x) = (x; \sum_{i}g_{i}e_{i})$ we can send it to
\[ \sum_{i = 1}^{n}g_{i}dx_{i} \]
to get a 1-form. So, we aim to show
\[ d \circ \imath_{S} = \imath_{V} \circ \operatorname{grad}. \]
\begin{proof}
  Clearly, almost by definition,
  \[ d \circ \imath_{S} = \sum_{i}f_{i}dx_{i} \]
  as this is just $df$ itself.
  Then, we notice that
  \begin{align*}
    (\imath_{V} \circ \operatorname{grad})f(x) &= \imath_{V}(x; \sum_{i}f_{i}(x)e_{i})\\
    &= \sum_{i}f_{i}dx_{i}.
  \end{align*}
  Since these are naturally equal, this square commutes.
\end{proof}

\underline{Step 2 (Bottom Square)}:
Again, $A \subset \RR[n]$ and we define for a vector field $F(x) = (x; \sum_{i}f_{i}e_{i})$ the operations:
\[ \div F = \sum_{i}D_{i}f_{i} \]
and
\[ \alpha F  = \sum_{i}\pars{-1}^{i - 1} f_{i} \bigwedge_{k \ne i}dx_{k}. \]
We also define the map $\beta$ from scalar fields as $f \mapsto f \bigwedge_{i} dx_{i}$.
So, we must show
\[ \beta \circ \div = d \circ \alpha. \]
\begin{proof}
  On the left hand side, we note
  \begin{align*}
    (\beta \circ \div)F &= \beta\pars*{\sum_{i}D_{i}f_{i}}\\
    &= \pars*{\sum_{i} D_{i}f_{i}} \bigwedge_{k} dx_{k} = \sum_{i} D_{i}f_{i} \bigwedge_{k} dx_{k}.
  \end{align*}
  On the right side, we find
  \begin{align*}
    (d \circ \alpha)F &= d\pars*{\sum_{i}\pars{-1}^{i - 1}f_{i} \bigwedge_{k \ne i} dx_{k}}\\
    &= \sum_{i}\pars{-1}^{i - 1}df_{i} \bigwedge_{k \ne i}dx_{k}.
  \end{align*}
  Note that $df_{i} = \sum_{j = 1}^{n}\frac{\partial f_{i}}{\partial x_{j}} dx_{j}$, and so in the sum above every term cancels except for when $j = i$,
  since those differential forms are already present in the big wedge.
  For each term, we must shift the new $dx_{i}$ over $i - 1$ times, and so the leading $-1$-power terms all go to positive 1.
  So, we find that this is equal to (with the substitution of notation that $\partial f_{i}/\partial x_{j} = D_{j}f_{i}$)
  \[ \sum_{i}D_{i}f_{i} \bigwedge_{k}dx_{k} \]
  as desired.
\end{proof}

\underline{Step 3 (Middle Square)}:
Now, we only work in $\RR[3]$, and use the operator
\[ \curl F(x) = (x; (D_{2}f_{3} - D_{3}f_{2})e_{1} + (D_{3}f_{1} - D_{1}f_{3})e_{2} + (D_{1}f_{2} - D_{2}f_{1})e_{3}) \]
and our $\alpha$ from before. Thus, our final goal is to show
\[ d \circ \imath_{V} = \alpha \circ \curl. \]
\begin{proof}
  To look at the left hand side first, we note that
  \[ (d \circ \imath_{V})F(x) = d\sum_{i}f_{i}dx_{i} = \sum_{i}df_{i}dx_{i} \]
  and so we may expand this as
  \[ df_{1}dx_{1} + df_{2}dx_{2} + df_{3}dx_{3}. \]
  Each term post-differential will only have 2 terms since the part corresponding to each $dx_{i}$ will go to zero by getting $dx_{i} \wedge dx_{i}$.
  For the first term, we get
  \[ df_{1}dx_{1} = D_{2}f_{1} dx_{2} \wedge dx_{1} + D_{3}f_{1} dx_{3} \wedge dx_{1} = - D_{2}f_{1} dx_{1} \wedge dx_{2} - D_{3}f_{1} dx_{1} \wedge dx_{3}. \]
  Doing similar expansions on the $f_{2}$ and $f_{3}$ terms, we get
  \begin{align*}
    df_{2}dx_{2} &= D_{1}f_{2} dx_{1} \wedge dx_{2} - D_{3}f_{2} dx_{2} \wedge dx_{3}\\
    df_{3}dx_{3} &= D_{1}f_{3} dx_{1} \wedge dx_{3} + D_{2}f_{3} dx_{2} \wedge dx_{3}.
  \end{align*}
  Grouping like terms by their differentials, we get
  \begin{equation}\label{eq:1} (D_{2}f_{3} - D_{3}f_{2}) dx_{2} \wedge dx_{3} + (D_{1}f_{3} - D_{3}f_{1}) dx_{1} \wedge dx_{3} + (D_{1}f_{2} - D_{2}f_{1}) dx_{1} \wedge dx_{2}. \end{equation}
  Turning to the right-hand side, we find (calling the resulting ``curled'' components $g_{i}$)
  \begin{align*}
    (\alpha \circ \curl)F(x) &= \alpha(x; (D_{2}f_{3} - D_{3}f_{2})e_{1} + (D_{3}f_{1} - D_{1}f_{3})e_{2} + (D_{1}f_{2} - D_{2}f_{1})e_{3})\\
                    &= \sum_{i = 1}^{3} \pars{-1}^{i - 1} g_{i} \bigwedge_{k \ne i}dx_{k}.
  \end{align*}
  Knowing our bound is 3, we can manually expand this as
  \[ (D_{2}f_{3} - D_{3}f_{2}) dx_{2} \wedge dx_{3} - (D_{3}f_{1} - D_{1}f_{3}) dx_{1} \wedge dx_{3} + (D_{1}f_{2} - D_{2}f_{1}) dx_{1} \wedge dx_{2}. \]
  Noticing that the negation gets distributed on the second term gets us exactly~\eqref{eq:1}, we find the composition is equal and so the whole diagram commutes.
\end{proof}

\end{document}
