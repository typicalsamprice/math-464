% -*- compile-command: "latexmk -pdf document.tex" -*-
\documentclass{article}

\author{Sam Price}
\title{Orthogonal Group Is Manifold}

\newif\ifprinted%
\printedtrue%
% Most taken from: https://github.com/SeniorMars/dotfiles/blob/main/latex_template/preamble.tex
\usepackage{geometry}

\usepackage[english]{babel}
\usepackage[T1]{fontenc}
\usepackage[utf8]{inputenc}
\usepackage{palatino}

% Use the command \doublespacing if needed
\usepackage{setspace}

\geometry{a4paper, margin=1in}

\usepackage{mathtools}
\usepackage{amssymb,amsfonts,amsthm,amsmath}

\usepackage{xfrac}
\usepackage[makeroom]{cancel}

% Use \begin{enumerate}[start=x,label={Q\arabic*)}] for example
\usepackage{enumitem}

\usepackage{xcolor}

\usepackage{nameref}

% Important options
% in envs, use options like [baseline=x] to center on row x (or special opts t/[c]/b without 'baseline')
\usepackage{nicematrix}
\NiceMatrixOptions{cell-space-limits = 1pt}

\usepackage{booktabs}

\usepackage{tikz}
\usepackage{tikz-cd}
\usepackage{tikzsymbols}

\usepackage{pdfpages}

\usepackage[most,many,breakable]{tcolorbox}

\setlength{\parindent}{1cm}

\DeclarePairedDelimiter{\abs}{\lvert}{\rvert}

%%%%%%%%%%%%%%%%%%%%%
%%% THEOREM BOXES %%%
%%%%%%%%%%%%%%%%%%%%%
\definecolor{thmbgcol}{HTML}{aec1f9}
\definecolor{thmhlcol}{HTML}{142c72}
\definecolor{corbgcol}{HTML}{b599f7}
\definecolor{corhlcol}{HTML}{2d1760}
\definecolor{propbgcol}{HTML}{c9f7aa}
\definecolor{prophlcol}{HTML}{1e631a}
\definecolor{exbgcol}{HTML}{f7c479}
\definecolor{exhlcol}{HTML}{604419}

\definecolor{qheadcol}{HTML}{182959}

\tcbuselibrary{theorems,skins,hooks}
\newtcbtheorem[number within = section]{theorem}{Theorem}{
  enhanced, breakable, colback = thmbgcol!25,
  frame hidden, boxrule = 0sp, borderline west = {2pt}{0pt}{thmhlcol},
  sharp corners, detach title, before upper = \tcbtitle\par\smallskip,
  coltitle = thmhlcol, fonttitle = \bfseries,
  description font = \mdseries, separator sign none, segmentation style = {solid, thmhlcol}
}{th}

\newtcbtheorem[number within = section]{corollary}{Corollary}{
  enhanced, breakable, colback = corbgcol!25,
  frame hidden, boxrule = 0sp, borderline west = {2pt}{0pt}{corhlcol},
  sharp corners, detach title, before upper = \tcbtitle\par\smallskip,
  coltitle = corhlcol, fonttitle = \bfseries,
  description font = \mdseries, separator sign none, segmentation style = {solid, corhlcol}
}{cor}

\newtcbtheorem[number within = section]{proposition}{Proposition}{
  enhanced, breakable, colback = propbgcol!25,
  frame hidden, boxrule = 0sp, borderline west = {2pt}{0pt}{prophlcol},
  sharp corners, detach title, before upper = \tcbtitle\par\smallskip,
  coltitle = prophlcol, fonttitle = \bfseries,
  description font = \mdseries, separator sign none, segmentation style = {solid, prophlcol}
}{prop}

\newtcbtheorem[number within = section]{example}{Example}{
  enhanced, breakable, colback = exbgcol!25,
  frame hidden, boxrule = 0sp, borderline west = {2pt}{0pt}{exhlcol},
  sharp corners, detach title, before upper = \tcbtitle\par\smallskip,
  coltitle = exhlcol, fonttitle = \bfseries,
  description font = \mdseries, separator sign none, segmentation style = {solid, exhlcol}
}{prop}

\newtcbtheorem[number within = section]{definition}{Definition}{
  enhanced, breakable, colback = red!10,
  frame hidden, boxrule = 0sp, borderline west = {2pt}{0pt}{red!50!black},
  sharp corners, detach title, before upper = \tcbtitle\par\smallskip,
  coltitle = red!50!black, fonttitle = \bfseries,
  description font = \mdseries, separator sign none, segmentation style = {solid, exhlcol}
}{def}

\makeatletter
\newtcbtheorem{question}{Question}{enhanced,
    breakable,
    colback=white,
    colframe=qheadcol,
    attach boxed title to top left={yshift*=-\tcboxedtitleheight},
    fonttitle=\bfseries,
    title={#2},
    boxed title size=title,
    boxed title style={%
            sharp corners,
            rounded corners=northwest,
            colback=qheadcol,
            boxrule=0pt,
        },
    underlay boxed title={%
            \path[fill=tcbcolframe] (title.south west)--(title.south east)
            to[out=0, in=180] ([xshift=5mm]title.east)--
            (title.center-|frame.east)
            [rounded corners=\kvtcb@arc] |-
            (frame.north) -| cycle;
        },
    #1
}{def}
\makeatother


\begin{document}

\maketitle

\begin{theorem}{}{}
  Let $f \from \RR[n + k] \to \RR[n]$ be $C^{r}$.
  Let $M = f\inv(0) \ne \emptyset$.
  If $\rank Df = n$ on all of $M$, then $M$ is a $k$-manifold without boundary in $\RR[n + k]$.
\end{theorem}
\begin{proof}
  Already proven.
\end{proof}


\begin{theorem}{}{}
  Show that, if $\Orth(3)$ is the group of $3 \times 3$ matrices of real numbers of determinant $\pm 1$ that
  \begin{enumerate}[start=1,label={\arabic*\rparen}]
    \item There is a smooth function $f \from \RR[9] \to \RR[6]$ so that $f\inv(0) = \Orth(3)$.
    \item $\Orth(3)$ is a compact 3-manifold without boundary.
  \end{enumerate}
\end{theorem}

\begin{proof}
  We want to use what we know about orthogonal matrices: their columns are of course orthonormal.
  So, we define a smooth function on $[c_{1}\ c_{2}\ c_{3}]$, the columns of a matrix in $M_{3}(\RR)$.
  \begin{align*}
    f_{1} &= c_{1} \cdot c_{2}\\
    f_{2} &= c_{1} \cdot c_{3}\\
    f_{3} &= c_{2} \cdot c_{3}.
  \end{align*}
  This takes care of the orthogonality; to consider the normality, we craft:
  \begin{align*}
    f_{4} &= c_{1} \cdot c_{1} - 1\\
    f_{5} &= c_{2} \cdot c_{2} - 1\\
    f_{6} &= c_{3} \cdot c_{3} - 1.
  \end{align*}
  So, by construction, a matrix $A = [c_{1}\ c_{2}\ c_{3}]$ gives $f(A) = 0$ if and only if $A \in \Orth(3)$.
  It is also clear $f$ is smooth, as all of the components are polynomial in the elements of $A$.

  To show $\Orth(3)$ is a compact 3-manifold, we use the previous theorem.
  Looking at $Df$, we see that (abusing notation of columns versus 3-tuples of real numbers):
  \begin{equation*}
    Df(c_{1}, c_{2}, c_{3}) = \begin{bmatrix}
      c_{2} & c_{1} & 0\\
      c_{3} & 0 & c_{1}\\
      0 & c_{3} & c_{2}\\
      2c_{1} & 0 & 0\\
      0 & 2c_{2} & 0\\
      0 & 0 & 2c_{3}
    \end{bmatrix}
  \end{equation*}
  Inside $\Orth(3)$, these $c_{i}$ are necessarily independent.
  It follows these rows are also independent ($A^{T} \in \Orth(3)$ as well),
  and so $\Orth(3)$ is a 3-manifold without boundary in $\RR[9]$.
  To verify compactness, note that each $a_{i} \in A \in \Orth(3)$ has at most magnitude 1.
  This means that $\Orth(3) \subset \set{ x \in \RR[9] \mid \norm{x}_{\sup} \le 1 }$ and is therefore bounded.
  Indeed, $\Orth(3) = f\inv(0)$ and so must be closed as well. By Heine-Borel, it is compact.
\end{proof}

\end{document}
