% -*- compile-command: "latexmk -pdf document.tex" -*-
\documentclass[12pt]{article}

\author{Sam Price}
\title{Reflection 10: CRUNCH Time, brought to you by Nestl\'e}

\newif\ifprinted%
%\printedtrue%
% Most taken from: https://github.com/SeniorMars/dotfiles/blob/main/latex_template/preamble.tex
\usepackage{geometry}

\usepackage[english]{babel}
\usepackage[T1]{fontenc}
\usepackage[utf8]{inputenc}
\usepackage{palatino}

% Use the command \doublespacing if needed
\usepackage{setspace}

\geometry{a4paper, margin=1in}

\usepackage{mathtools}
\usepackage{amssymb,amsfonts,amsthm,amsmath}

\usepackage{xfrac}
\usepackage[makeroom]{cancel}

% Use \begin{enumerate}[start=x,label={Q\arabic*)}] for example
\usepackage{enumitem}

\usepackage{xcolor}

\usepackage{nameref}

% Important options
% in envs, use options like [baseline=x] to center on row x (or special opts t/[c]/b without 'baseline')
\usepackage{nicematrix}
\NiceMatrixOptions{cell-space-limits = 1pt}

\usepackage{booktabs}

\usepackage{tikz}
\usepackage{tikz-cd}
\usepackage{tikzsymbols}

\usepackage{pdfpages}

\usepackage[most,many,breakable]{tcolorbox}

\setlength{\parindent}{1cm}

\DeclarePairedDelimiter{\abs}{\lvert}{\rvert}

%%%%%%%%%%%%%%%%%%%%%
%%% THEOREM BOXES %%%
%%%%%%%%%%%%%%%%%%%%%
\definecolor{thmbgcol}{HTML}{aec1f9}
\definecolor{thmhlcol}{HTML}{142c72}
\definecolor{corbgcol}{HTML}{b599f7}
\definecolor{corhlcol}{HTML}{2d1760}
\definecolor{propbgcol}{HTML}{c9f7aa}
\definecolor{prophlcol}{HTML}{1e631a}
\definecolor{exbgcol}{HTML}{f7c479}
\definecolor{exhlcol}{HTML}{604419}

\definecolor{qheadcol}{HTML}{182959}

\tcbuselibrary{theorems,skins,hooks}
\newtcbtheorem[number within = section]{theorem}{Theorem}{
  enhanced, breakable, colback = thmbgcol!25,
  frame hidden, boxrule = 0sp, borderline west = {2pt}{0pt}{thmhlcol},
  sharp corners, detach title, before upper = \tcbtitle\par\smallskip,
  coltitle = thmhlcol, fonttitle = \bfseries,
  description font = \mdseries, separator sign none, segmentation style = {solid, thmhlcol}
}{th}

\newtcbtheorem[number within = section]{corollary}{Corollary}{
  enhanced, breakable, colback = corbgcol!25,
  frame hidden, boxrule = 0sp, borderline west = {2pt}{0pt}{corhlcol},
  sharp corners, detach title, before upper = \tcbtitle\par\smallskip,
  coltitle = corhlcol, fonttitle = \bfseries,
  description font = \mdseries, separator sign none, segmentation style = {solid, corhlcol}
}{cor}

\newtcbtheorem[number within = section]{proposition}{Proposition}{
  enhanced, breakable, colback = propbgcol!25,
  frame hidden, boxrule = 0sp, borderline west = {2pt}{0pt}{prophlcol},
  sharp corners, detach title, before upper = \tcbtitle\par\smallskip,
  coltitle = prophlcol, fonttitle = \bfseries,
  description font = \mdseries, separator sign none, segmentation style = {solid, prophlcol}
}{prop}

\newtcbtheorem[number within = section]{example}{Example}{
  enhanced, breakable, colback = exbgcol!25,
  frame hidden, boxrule = 0sp, borderline west = {2pt}{0pt}{exhlcol},
  sharp corners, detach title, before upper = \tcbtitle\par\smallskip,
  coltitle = exhlcol, fonttitle = \bfseries,
  description font = \mdseries, separator sign none, segmentation style = {solid, exhlcol}
}{prop}

\newtcbtheorem[number within = section]{definition}{Definition}{
  enhanced, breakable, colback = red!10,
  frame hidden, boxrule = 0sp, borderline west = {2pt}{0pt}{red!50!black},
  sharp corners, detach title, before upper = \tcbtitle\par\smallskip,
  coltitle = red!50!black, fonttitle = \bfseries,
  description font = \mdseries, separator sign none, segmentation style = {solid, exhlcol}
}{def}

\makeatletter
\newtcbtheorem{question}{Question}{enhanced,
    breakable,
    colback=white,
    colframe=qheadcol,
    attach boxed title to top left={yshift*=-\tcboxedtitleheight},
    fonttitle=\bfseries,
    title={#2},
    boxed title size=title,
    boxed title style={%
            sharp corners,
            rounded corners=northwest,
            colback=qheadcol,
            boxrule=0pt,
        },
    underlay boxed title={%
            \path[fill=tcbcolframe] (title.south west)--(title.south east)
            to[out=0, in=180] ([xshift=5mm]title.east)--
            (title.center-|frame.east)
            [rounded corners=\kvtcb@arc] |-
            (frame.north) -| cycle;
        },
    #1
}{def}
\makeatother


\begin{document}

\maketitle

All the math I've done has been in the past four to five days.
It has been fairly productive and been a good ol' learning time.
The first problem I tidied up was the exercise showing that a crossed area can't be homeomorphic to $\RR$ (or any interval)
and while most of the work was mostly complete, I rewrote most of it because I wasn't happy with it.
The part I fixed (I hope this was the one at least) is just explaining why the four disconnected components can't map bijectively
onto the two disconnected components of the punctured line. It comes down to missing at least two points, or overlap happening and so it's impossible.
I'll also chunk through the algebra of showing $\rank D\det = 1$ so that the exercise I presented today is more complete.

For my roadmap, I plan on doing (provided they take long enough/feel like there is some new insight I gain):
\begin{itemize}
  \item Exercise 5.24.5. Part (a) is to find some $f \from \RR[9] \to \RR$ so that $f\inv(0) = \Orth_{3}(\RR)$.
        Then, Part (b) is to show that $\Orth_{3}(\RR)$ is a compact manifold without boundary.
  \item Finish showing the determinant equivalence if I have the time.
        Fall break I imagine I will though, since I will have my laptop and a couple notebooks only.
  \item Ex. 4.20.1: If $h$ is an orthogonal transformation, then $h$ carries every orthonormal set to another orthonormal set.
\end{itemize}

To try to get differential forms more: when do we determine which point $p \in \M$ that $T_{p}\M$ is being derived from?
I guess more sensibly, we have the notation $\omega$ and $\omega_{p}$ separately for specifying $p$. At what point does this distinction matter
and change what $\omega$ is doing? Is $\omega$ itself on $T\M$ as a whole (or $T^{\ast}\M$ and $T^{\ast}_{p}\M$ respectively?)
and then we drill down to $T_{p}\M$ itself and therefore we get $\omega_{p}$?
I'm going to read a few sections of the textbook tomorrow to see if I can glean any more information.

I don't have any other major hangups as of now, but I have a feeling I should be taking notes quicker rather than trying to get it
typed -\ I can go back and tidy up or transfer from pen to screen later (and probably should for review anyways).
That will be my big change for Thursday and beyond.


\end{document}
