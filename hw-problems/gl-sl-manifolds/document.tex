% -*- compile-command: "latexmk -pdf document.tex" -*-
\documentclass{article}

\author{Sam Price}
\title{$\GL, \SL$ are manifolds.}

\newif\ifprinted%
\printedtrue%
% Most taken from: https://github.com/SeniorMars/dotfiles/blob/main/latex_template/preamble.tex
\usepackage{geometry}

\usepackage[english]{babel}
\usepackage[T1]{fontenc}
\usepackage[utf8]{inputenc}
\usepackage{palatino}

% Use the command \doublespacing if needed
\usepackage{setspace}

\geometry{a4paper, margin=1in}

\usepackage{mathtools}
\usepackage{amssymb,amsfonts,amsthm,amsmath}

\usepackage{xfrac}
\usepackage[makeroom]{cancel}

% Use \begin{enumerate}[start=x,label={Q\arabic*)}] for example
\usepackage{enumitem}

\usepackage{xcolor}

\usepackage{nameref}

% Important options
% in envs, use options like [baseline=x] to center on row x (or special opts t/[c]/b without 'baseline')
\usepackage{nicematrix}
\NiceMatrixOptions{cell-space-limits = 1pt}

\usepackage{booktabs}

\usepackage{tikz}
\usepackage{tikz-cd}
\usepackage{tikzsymbols}

\usepackage{pdfpages}

\usepackage[most,many,breakable]{tcolorbox}

\setlength{\parindent}{1cm}

\DeclarePairedDelimiter{\abs}{\lvert}{\rvert}

%%%%%%%%%%%%%%%%%%%%%
%%% THEOREM BOXES %%%
%%%%%%%%%%%%%%%%%%%%%
\definecolor{thmbgcol}{HTML}{aec1f9}
\definecolor{thmhlcol}{HTML}{142c72}
\definecolor{corbgcol}{HTML}{b599f7}
\definecolor{corhlcol}{HTML}{2d1760}
\definecolor{propbgcol}{HTML}{c9f7aa}
\definecolor{prophlcol}{HTML}{1e631a}
\definecolor{exbgcol}{HTML}{f7c479}
\definecolor{exhlcol}{HTML}{604419}

\definecolor{qheadcol}{HTML}{182959}

\tcbuselibrary{theorems,skins,hooks}
\newtcbtheorem[number within = section]{theorem}{Theorem}{
  enhanced, breakable, colback = thmbgcol!25,
  frame hidden, boxrule = 0sp, borderline west = {2pt}{0pt}{thmhlcol},
  sharp corners, detach title, before upper = \tcbtitle\par\smallskip,
  coltitle = thmhlcol, fonttitle = \bfseries,
  description font = \mdseries, separator sign none, segmentation style = {solid, thmhlcol}
}{th}

\newtcbtheorem[number within = section]{corollary}{Corollary}{
  enhanced, breakable, colback = corbgcol!25,
  frame hidden, boxrule = 0sp, borderline west = {2pt}{0pt}{corhlcol},
  sharp corners, detach title, before upper = \tcbtitle\par\smallskip,
  coltitle = corhlcol, fonttitle = \bfseries,
  description font = \mdseries, separator sign none, segmentation style = {solid, corhlcol}
}{cor}

\newtcbtheorem[number within = section]{proposition}{Proposition}{
  enhanced, breakable, colback = propbgcol!25,
  frame hidden, boxrule = 0sp, borderline west = {2pt}{0pt}{prophlcol},
  sharp corners, detach title, before upper = \tcbtitle\par\smallskip,
  coltitle = prophlcol, fonttitle = \bfseries,
  description font = \mdseries, separator sign none, segmentation style = {solid, prophlcol}
}{prop}

\newtcbtheorem[number within = section]{example}{Example}{
  enhanced, breakable, colback = exbgcol!25,
  frame hidden, boxrule = 0sp, borderline west = {2pt}{0pt}{exhlcol},
  sharp corners, detach title, before upper = \tcbtitle\par\smallskip,
  coltitle = exhlcol, fonttitle = \bfseries,
  description font = \mdseries, separator sign none, segmentation style = {solid, exhlcol}
}{prop}

\newtcbtheorem[number within = section]{definition}{Definition}{
  enhanced, breakable, colback = red!10,
  frame hidden, boxrule = 0sp, borderline west = {2pt}{0pt}{red!50!black},
  sharp corners, detach title, before upper = \tcbtitle\par\smallskip,
  coltitle = red!50!black, fonttitle = \bfseries,
  description font = \mdseries, separator sign none, segmentation style = {solid, exhlcol}
}{def}

\makeatletter
\newtcbtheorem{question}{Question}{enhanced,
    breakable,
    colback=white,
    colframe=qheadcol,
    attach boxed title to top left={yshift*=-\tcboxedtitleheight},
    fonttitle=\bfseries,
    title={#2},
    boxed title size=title,
    boxed title style={%
            sharp corners,
            rounded corners=northwest,
            colback=qheadcol,
            boxrule=0pt,
        },
    underlay boxed title={%
            \path[fill=tcbcolframe] (title.south west)--(title.south east)
            to[out=0, in=180] ([xshift=5mm]title.east)--
            (title.center-|frame.east)
            [rounded corners=\kvtcb@arc] |-
            (frame.north) -| cycle;
        },
    #1
}{def}
\makeatother


\begin{document}

\maketitle

First, let us consider $\GL_{n}(\RR)$.
We can consider each $A \in \GL_{n}$ as an element of $\RR[n^{2}]$ by ``flattening'' the rows (or columns) into one vector.
Of course, we know that
\[ \underline{\GL_{n}(\RR) = \set{ A \in \mathcal{M}_{n}(\RR) \mid \det A \ne 0 }} \]
Because $\det \from \RR[n^{2}] \to \RR$ is a polynomial, it must be continuous.
Therefore, $\GL_{n} = \det\inv(\RR - \set{0})$ must be open as well.
Because $\RR[n^{2}]$ is a manifold, it is charted.
The restrictions of these charts to open subsets (intersections of $\RR[n^{2}]$ and $\GL$) of their domains keeps their desired properties (diffeomorphic, etc.) in check.
Hence, $\GL_{n}$ is a manifold (in fact, this reasoning shows if $M$ is a manifold and $U \subset M$ is open, then $U$ is a manifold as well.)

\begin{center}
  \underline{Looking to $\SL_{n} = \set{ A \in \mathcal{M}_{n} \mid \det A = 1 }$}
\end{center}
Define $F(A) = \det A - 1$. Then, $\SL_{n} = F\inv(0)$.
Also, $D\det$ has constant rank 1 on $\SL_{n}$ due to its nonzero determinant (it is 1 specifically, but it only requires nonzero-osity).
Therefore, by the theorem below, it is a manifold that is the zero locus of the function $F \from \RR[1 + n^{2} - 1] \to \RR$ and so
is of dimension $n^{2} - 1$.

\begin{theorem}{Munkres 5.24, Part of Theorem in Exercise 2}{}
  Let $f \from \RR[n + k] \to \RR[n]$ be of class $C^{r}$. Let $M = f\inv(0)$, and assume $M$ nonempty and that $\rank Df = n$ for all $x \in M$.
  Then, $M$ is a $k$-manifold in $\RR[n + k]$.

  [Rest of theorem omitted, as it is not used]
\end{theorem}
\begin{proof}
  Let $(p, q) \in M$ so that $p \in \RR[k]$ and $q \in \RR[n]$.
  Then, there is some $g \from U \to \RR[n]$ for an open $U \subset \RR[k]$ containing $p$ so that
  \[ g(p) = q,\quad f(x, g(x)) = 0 \]
  for all $x \in U$. We may further restrict this (since $g$ is $C^{r}$) so that the inverse function theorem may be applied;
  therefore, $g$ is a chart on $M$ and as such $M$ is a $k$-manifold.
\end{proof}

\begin{proposition}{Determinant is nonsingular}{}
  The function $\det \from \GL_{n} \to \RR[\times]$ has rank 1 Jacobian everywhere.
\end{proposition}
\begin{proof}
  We know that since this is a $1 \times n^{2}$ matrix, it can only have rank zero when it \emph{is} the zero row matrix.
  Consider $a_{11}$ in a given matrix $A$ that satisfies this. Its entry in $D\det$ is the determinant of the minor related to itself.
  We only now need consider this first row in the matrix: every single one must be zero to let $D\det = 0$ get close to true.
  This means that
  \[ \det A = \sum_{i = 1}^{n}a_{1,n}M^{\ast} = 0 \]
  where $M^{\ast}$ is the (determinant of the) minor determined by removing the first row and $i$th column of the input matrix.
  However, this is not possible as we assumed $A$ is in $\GL_{n}$ and so has nonzero determinant.
\end{proof}

\end{document}
