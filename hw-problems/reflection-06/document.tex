% -*- compile-command: "latexmk -pdf document.tex" -*-
\documentclass[12pt]{article}

\usepackage{graphicx}

\author{Sam Price}
\title{\reflectbox{Reflection Six}}

\newif\ifprinted%
%\printedtrue%
% Most taken from: https://github.com/SeniorMars/dotfiles/blob/main/latex_template/preamble.tex
\usepackage{geometry}

\usepackage[english]{babel}
\usepackage[T1]{fontenc}
\usepackage[utf8]{inputenc}
\usepackage{palatino}

% Use the command \doublespacing if needed
\usepackage{setspace}

\geometry{a4paper, margin=1in}

\usepackage{mathtools}
\usepackage{amssymb,amsfonts,amsthm,amsmath}

\usepackage{xfrac}
\usepackage[makeroom]{cancel}

% Use \begin{enumerate}[start=x,label={Q\arabic*)}] for example
\usepackage{enumitem}

\usepackage{xcolor}

\usepackage{nameref}

% Important options
% in envs, use options like [baseline=x] to center on row x (or special opts t/[c]/b without 'baseline')
\usepackage{nicematrix}
\NiceMatrixOptions{cell-space-limits = 1pt}

\usepackage{booktabs}

\usepackage{tikz}
\usepackage{tikz-cd}
\usepackage{tikzsymbols}

\usepackage{pdfpages}

\usepackage[most,many,breakable]{tcolorbox}

\setlength{\parindent}{1cm}

\DeclarePairedDelimiter{\abs}{\lvert}{\rvert}

%%%%%%%%%%%%%%%%%%%%%
%%% THEOREM BOXES %%%
%%%%%%%%%%%%%%%%%%%%%
\definecolor{thmbgcol}{HTML}{aec1f9}
\definecolor{thmhlcol}{HTML}{142c72}
\definecolor{corbgcol}{HTML}{b599f7}
\definecolor{corhlcol}{HTML}{2d1760}
\definecolor{propbgcol}{HTML}{c9f7aa}
\definecolor{prophlcol}{HTML}{1e631a}
\definecolor{exbgcol}{HTML}{f7c479}
\definecolor{exhlcol}{HTML}{604419}

\definecolor{qheadcol}{HTML}{182959}

\tcbuselibrary{theorems,skins,hooks}
\newtcbtheorem[number within = section]{theorem}{Theorem}{
  enhanced, breakable, colback = thmbgcol!25,
  frame hidden, boxrule = 0sp, borderline west = {2pt}{0pt}{thmhlcol},
  sharp corners, detach title, before upper = \tcbtitle\par\smallskip,
  coltitle = thmhlcol, fonttitle = \bfseries,
  description font = \mdseries, separator sign none, segmentation style = {solid, thmhlcol}
}{th}

\newtcbtheorem[number within = section]{corollary}{Corollary}{
  enhanced, breakable, colback = corbgcol!25,
  frame hidden, boxrule = 0sp, borderline west = {2pt}{0pt}{corhlcol},
  sharp corners, detach title, before upper = \tcbtitle\par\smallskip,
  coltitle = corhlcol, fonttitle = \bfseries,
  description font = \mdseries, separator sign none, segmentation style = {solid, corhlcol}
}{cor}

\newtcbtheorem[number within = section]{proposition}{Proposition}{
  enhanced, breakable, colback = propbgcol!25,
  frame hidden, boxrule = 0sp, borderline west = {2pt}{0pt}{prophlcol},
  sharp corners, detach title, before upper = \tcbtitle\par\smallskip,
  coltitle = prophlcol, fonttitle = \bfseries,
  description font = \mdseries, separator sign none, segmentation style = {solid, prophlcol}
}{prop}

\newtcbtheorem[number within = section]{example}{Example}{
  enhanced, breakable, colback = exbgcol!25,
  frame hidden, boxrule = 0sp, borderline west = {2pt}{0pt}{exhlcol},
  sharp corners, detach title, before upper = \tcbtitle\par\smallskip,
  coltitle = exhlcol, fonttitle = \bfseries,
  description font = \mdseries, separator sign none, segmentation style = {solid, exhlcol}
}{prop}

\newtcbtheorem[number within = section]{definition}{Definition}{
  enhanced, breakable, colback = red!10,
  frame hidden, boxrule = 0sp, borderline west = {2pt}{0pt}{red!50!black},
  sharp corners, detach title, before upper = \tcbtitle\par\smallskip,
  coltitle = red!50!black, fonttitle = \bfseries,
  description font = \mdseries, separator sign none, segmentation style = {solid, exhlcol}
}{def}

\makeatletter
\newtcbtheorem{question}{Question}{enhanced,
    breakable,
    colback=white,
    colframe=qheadcol,
    attach boxed title to top left={yshift*=-\tcboxedtitleheight},
    fonttitle=\bfseries,
    title={#2},
    boxed title size=title,
    boxed title style={%
            sharp corners,
            rounded corners=northwest,
            colback=qheadcol,
            boxrule=0pt,
        },
    underlay boxed title={%
            \path[fill=tcbcolframe] (title.south west)--(title.south east)
            to[out=0, in=180] ([xshift=5mm]title.east)--
            (title.center-|frame.east)
            [rounded corners=\kvtcb@arc] |-
            (frame.north) -| cycle;
        },
    #1
}{def}
\makeatother


\begin{document}

\maketitle

This week has been actually quite productive in terms of math.
I'll start with the success stories.
I completed 5 exercises in the book!
Most of them didn't take long at all and were scattered from differentiation to
change-of-variables chapters, so I won't plan on presenting those ever.
The problem that took the longest (and that I actually wrote up)
was in the Implicit Function Theorem section, as number 5.
Once I figured out the trick of creating $h(x) = [f(x) g(x)]$
it was actually quite simple since the dimensions lined up correctly.
After that, it was time for the actual week, and so I figured out the trick quite nicely
for the problem you gave us! I wasn't entirely sure about the generalization, but figured
if it \emph{locally} looked like the axes in $\RR[2]$, then one could translate them separately
and still have it make some sort of sense. In any case, it didn't matter. All that I have left
to do is explain why $f$ being bijective implies $f\inv(I) \cap f\inv(J) = f\inv(I \cap J)$.

In terms of the struggles this past week, it mostly comes down to things being quite gnarly
or notation clogging up intuition. When you were presenting a proof of Theorem 16.5
(page 141, PDF page 154) from the book it definitely feels strange since, as mentioned, there
are \emph{so} few cases any explicit partition of unity would be practical to do computations
with. It's not hard to follow the proof beyond that aspect at least.

Beyond that, I think it will take a little bit to adapt to the definition(s) of manifolds we've
talked about so far. My exposure to them at this point has basically only been the perhaps cliche
tagline of ``they look like Euclidean space'' when on a manifold. This is sort of what I meant
when talking about projective varieties as 2-dimensional. They clearly aren't, they live in
``three dimensions'' because they \emph{need} that space, but they locally look like $\RR[2]$
just bent in a fancy way to be non-flat. This is what the whole $U \to V \cap X$ notation is suppose
to exemplify though, so I think it will just take getting used to shifting to a more formalized
notion to gain a greater understanding.

For planning, I want to at least think about the 2-torus problem you gave us today as well.
My approach for that will be to choose some specfic values for $r, R$ and then determine
a couple charts for it. For the minimum, it intuitively feels like 8 would be the answer.
That comes from each ``sliced'' part looking similar to the body of a cylinder,
which needs 2 charts in some sense because it's like $S^{1}$. Since each quadrant would have these
issues, it would be $4 \times 2 = 8$ charts to cover any $T^{2}$
(were you drawing $\Pi^{2}$ on the board? I could not read it and should have clarified).

For my presentation I'll try to whip something up in Beamer since I think it both
looks the nicest for math, and is something I should learn in general.
As a blueprint, I think it will be something like
\begin{itemize}
  \item What is a variety? Zero-sets of polynomials and a couple simple $\mathbb{A}^{2}$ examples.
        In this case, is it correct to describe affine space as a place similar to where
        we (as schoolkids) classically learned geometry, since there is distance but
        any given point or line has no meaning of ``distance'' from any supposed origin?
  \item Projective space, and visualizing $\PP[1]$ and $\PP[2]$.
        Talk about why adding a point to $\CC$ at infinity is taken
        as $\CC[2]/\negmedspace\sim$ instead of literally using $\CC \cup \set{\infty}$.
        From what I can understand this whole time, it comes down to when we extend to
        higher dimensions there is a need for different directions to infinity.

  \item Why are homogeneous polynomials the subject in this setting?
        Well, it's because of the relation we just endowed on $\CC[n]$.
        We'd really like $f(x) = f(y) = 0$ if $x \sim y$. Other values might not be the same,
        but that isn't really important since $f(x) = 0 \iff f(\lambda x) = 0$ for nonzero $\lambda$
        and those are the points making up our varieties.

  \item What is my goal with research?
        Describe the parameter space (correct terminology?) of cubic surfaces in $\PP[19]$
        and asking the question of how many of the 27 lines are real.

  \item Show some pictures, Clebsch is a given and a couple others.
\end{itemize}


\end{document}
