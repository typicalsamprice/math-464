% -*- compile-command: "latexmk -pdf document.tex" -*-
\documentclass[12pt]{article}

/home/sam/Git/latex-template/template.tex
\togglefalse{paper}

%%%%%%%%%%%%%%%%%%%%%
%%% THEOREM BOXES %%%
%%%%%%%%%%%%%%%%%%%%%
\usepackage[most,many,breakable]{tcolorbox}
\tcbuselibrary{theorems,skins,hooks}

% Stolen from: https://tex.stackexchange.com/a/330460
\makeatletter
\renewenvironment{proof}[1][\proofname]{\par
  \pushQED{\qed}%
  \normalfont \topsep6\p@\@plus6\p@\relax
  \trivlist
  \item[%
    \hskip\labelsep
    \normalfont\bfseries % was \itshape
    #1%
    \@addpunct{.}% remove this if you don't want punctuation
  ]\ignorespaces
}{%
  \popQED\endtrivlist\@endpefalse
}
\let\qed\relax % avoid a warning
\DeclareRobustCommand{\qed}{%
  \ifmmode \mathqed
  \else
    \leavevmode\unskip\penalty\@M\hbox{}\nobreak\hspace{.5em minus .1em}% was \hfill
    \hbox{\qedsymbol}%
  \fi
}
\makeatother

\ifprinted%
  \colorlet{thmbgcol}{white}
  \colorlet{lembgcol}{white}
  \colorlet{corbgcol}{white}
  \colorlet{propbgcol}{white}
  \colorlet{exbgcol}{white}
  \colorlet{defbgcol}{white}

  \colorlet{qheadcol}{black!20!white}
  \colorlet{qheadtxtcol}{black!90}

  \colorlet{exhlcol}{black}
  \colorlet{prophlcol}{black}
  \colorlet{corhlcol}{black}
  \colorlet{thmhlcol}{black}
  \colorlet{lemhlcol}{black}
  \colorlet{defhlcol}{black}
\else
  \definecolor{thmbgcol}{HTML}{aec1f9}
  \definecolor{corbgcol}{HTML}{b599f7}
  \definecolor{propbgcol}{HTML}{c9f7aa}
  \definecolor{exbgcol}{HTML}{f7c479}
  \colorlet{defbgcol}{red!7}

  \definecolor{qheadcol}{HTML}{182959}
  \colorlet{qheadtxtcol}{white}

  \definecolor{exhlcol}{HTML}{604419}
  \definecolor{prophlcol}{HTML}{1e631a}
  \definecolor{corhlcol}{HTML}{2d1760}
  \definecolor{thmhlcol}{HTML}{142c72}
  \colorlet{defhlcol}{red!50!black}

  \definecolor{lembgcol}{HTML}{e998f2}
  \definecolor{lemhlcol}{HTML}{791684}
\fi

\newtcbtheorem[number within = section]{theorem}{Theorem}{
  enhanced, breakable, colback = thmbgcol!20,
  frame hidden, boxrule = 0sp, borderline west = {2pt}{0pt}{thmhlcol},
  sharp corners, detach title, before upper = \tcbtitle\par\smallskip,
  coltitle = thmhlcol, fonttitle = \bfseries,
  description font = \mdseries, separator sign none, segmentation style = {solid, thmhlcol}
}{th}

\newtcbtheorem[number within = section]{lemma}{Lemma}{
  enhanced, breakable, colback = lembgcol!20,
  frame hidden, boxrule = 0sp, borderline west = {2pt}{0pt}{lemhlcol},
  sharp corners, detach title, before upper = \tcbtitle\par\smallskip,
  coltitle = lemhlcol, fonttitle = \bfseries,
  description font = \mdseries, separator sign none, segmentation style = {solid, lemhlcol}
}{lem}

\newtcbtheorem[number within = section]{corollary}{Corollary}{
  enhanced, breakable, colback = corbgcol!20,
  frame hidden, boxrule = 0sp, borderline west = {2pt}{0pt}{corhlcol},
  sharp corners, detach title, before upper = \tcbtitle\par\smallskip,
  coltitle = corhlcol, fonttitle = \bfseries,
  description font = \mdseries, separator sign none, segmentation style = {solid, corhlcol}
}{cor}

\newtcbtheorem[number within = section]{proposition}{Proposition}{
  enhanced, breakable, colback = propbgcol!25,
  frame hidden, boxrule = 0sp, borderline west = {2pt}{0pt}{prophlcol},
  sharp corners, detach title, before upper = \tcbtitle\par\smallskip,
  coltitle = prophlcol, fonttitle = \bfseries,
  description font = \mdseries, separator sign none, segmentation style = {solid, prophlcol}
}{prop}

\newtcbtheorem[number within = section]{example}{Example}{
  enhanced, breakable, colback = exbgcol!20,
  frame hidden, boxrule = 0sp, borderline west = {2pt}{0pt}{exhlcol},
  sharp corners, detach title, before upper = \tcbtitle\par\smallskip,
  coltitle = exhlcol, fonttitle = \bfseries,
  description font = \mdseries, separator sign none, segmentation style = {solid, exhlcol}
}{prop}

\newtcbtheorem[number within = section]{definition}{Definition}{
  enhanced, breakable, colback = defbgcol,
  frame hidden, boxrule = 0sp, borderline west = {2pt}{0pt}{defhlcol},
  sharp corners, detach title, before upper = \tcbtitle\par\smallskip,
  coltitle = defhlcol, fonttitle = \bfseries,
  description font = \mdseries, separator sign none, segmentation style = {solid, exhlcol}
}{def}

\makeatletter
\newtcbtheorem{question}{Question}{enhanced,
    breakable,
    colback=white,
    colframe=qheadcol,
    attach boxed title to top left={yshift*=-\tcboxedtitleheight},
    fonttitle=\bfseries,
    coltitle=qheadtxtcol,
    title={#2},
    boxed title size=title,
    boxed title style={%
            sharp corners,
            rounded corners=northwest,
            colback=qheadcol,
            boxrule=0pt,
        },
    underlay boxed title={%
            \path[fill=tcbcolframe] (title.south west)--(title.south east)
            to[out=0, in=180] ([xshift=5mm]title.east)--
            (title.center-|frame.east)
            [rounded corners=\kvtcb@arc] |-
            (frame.north) -| cycle;
        },
    #1
}{}
\makeatother


\author{Sam Price}
\date{}
\title{Reflection Three}

\begin{document}

\maketitle

This week was particularly interesting.
I didn't do \emph{any} work the entire weekend because I either worked for 12 hours a day or had
5 assignments due on Sunday that took priority because to the strict deadline.
There's a lot of good and meh this time around, but nothing too bad at all.
I'm getting more and more comfortable working with matrices as derivatives, and while my life
was slipping away in real-time I think I actually get how integration will work to some extent.
The only nitpick I talked myself through is that each $I_{i}$ is in $P_{i} \in P$.
You may have covered that, but again, I was on the verge of sleep towards the end.

The best part of this week in terms of math was obviously \emph{finally} realizing the trick to proving that $\RR[n]$ is complete.
The biggest trick was finding a way to actually \emph{use} the fact a sequence is Cauchy beyond my kernel of an
idea that would leverage boundedness.
I was spending so much time trying to prove Bolzano-Weierstrass but couldn't get past
the possibility of the component that determines the sup norm switching. I'll try to get it written up to
give to you during class on Thursday, although there's a possibility I finish writing it up before I leave my room tomorrow morning and can give it to you earlier.
This exercise also felt important to do because after I did a little bit of reading on the inverse function theorem
and the Banach fixed-point theorem, I realized it relied on the fact that $\RR[n]$ is complete to justify a sequence
being Cauchy and thus convergent.

In terms of work so far on my capstone proof, I don't have much besides a cursory glance at the big parts of it.
The sequential proof of Banach looks interesting, although I wonder if there is some argument you can make
that there is both at least one fixed point and at \emph{most} one fixed point.
The latter is obvious to me, and really the only reason I am considering such a path.
My thought process is mainly: take two distinct fixed points $x_{1}, x_{2}$.
Then, $\norm{f(x_{1}) - f(x_{2})} = \norm{x_{1} - x_{2}}$ and contradicts the fact $f$ is a contraction.
I have plenty of time (at present at least) so I am not going to worry too much yet.

My next exercise(s) that I want to present are twofold:
play around with the elliptic curve scenario and at least do part (a) of finding $\mathbb{V}(f_{\lm})$.
It seems at least interesting, and more difficult than what I've done so far.
The other part is chapter 3 ex. 6, the theorem declaring that $f \from Q \to \bbR$ is integrable if and only if
for every $\eps > 0$ there is a partition of mesh $\delta > 0$ that has the upper and lower sums within $\eps$ of
each other.

My greatest weakness right now is linear algebra still, and so my focus will definitely be on that cofactor
expansion demonstration you talked about today, to show that the determinant of that matrix
is ``easy'' and intuitive. For now, I think the best course of action is to try and make 355 obsolete.


\vspace{1em}
Fun and unrelated math...news? While in graph theory, we looked at the proof for a graph having an Eulerian circuit.
Clearly, using some hidden induction, the proof was valid for finite graphs. Thinking about it a bit more, is there some
really awful and hacky solution for infinite graphs (whose vertices are all of even degree) that can be made?
My main idea for it would be finding a constructed chain of subgraphs that are Eulerian and then using Zorn's lemma in some
fancy-shmancy way.

\end{document}
