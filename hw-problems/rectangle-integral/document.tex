% -*- compile-command: "latexmk -pdf document.tex" -*-
\documentclass{article}

\author{Sam Price}
\title{Additive Integration, The Hard Way}

\newif\ifprinted%
\printedtrue%
% Most taken from: https://github.com/SeniorMars/dotfiles/blob/main/latex_template/preamble.tex
\usepackage{geometry}

\usepackage[english]{babel}
\usepackage[T1]{fontenc}
\usepackage[utf8]{inputenc}
\usepackage{palatino}

% Use the command \doublespacing if needed
\usepackage{setspace}

\geometry{a4paper, margin=1in}

\usepackage{mathtools}
\usepackage{amssymb,amsfonts,amsthm,amsmath}

\usepackage{xfrac}
\usepackage[makeroom]{cancel}

% Use \begin{enumerate}[start=x,label={Q\arabic*)}] for example
\usepackage{enumitem}

\usepackage{xcolor}

\usepackage{nameref}

% Important options
% in envs, use options like [baseline=x] to center on row x (or special opts t/[c]/b without 'baseline')
\usepackage{nicematrix}
\NiceMatrixOptions{cell-space-limits = 1pt}

\usepackage{booktabs}

\usepackage{tikz}
\usepackage{tikz-cd}
\usepackage{tikzsymbols}

\usepackage{pdfpages}

\usepackage[most,many,breakable]{tcolorbox}

\setlength{\parindent}{1cm}

\DeclarePairedDelimiter{\abs}{\lvert}{\rvert}

%%%%%%%%%%%%%%%%%%%%%
%%% THEOREM BOXES %%%
%%%%%%%%%%%%%%%%%%%%%
\definecolor{thmbgcol}{HTML}{aec1f9}
\definecolor{thmhlcol}{HTML}{142c72}
\definecolor{corbgcol}{HTML}{b599f7}
\definecolor{corhlcol}{HTML}{2d1760}
\definecolor{propbgcol}{HTML}{c9f7aa}
\definecolor{prophlcol}{HTML}{1e631a}
\definecolor{exbgcol}{HTML}{f7c479}
\definecolor{exhlcol}{HTML}{604419}

\definecolor{qheadcol}{HTML}{182959}

\tcbuselibrary{theorems,skins,hooks}
\newtcbtheorem[number within = section]{theorem}{Theorem}{
  enhanced, breakable, colback = thmbgcol!25,
  frame hidden, boxrule = 0sp, borderline west = {2pt}{0pt}{thmhlcol},
  sharp corners, detach title, before upper = \tcbtitle\par\smallskip,
  coltitle = thmhlcol, fonttitle = \bfseries,
  description font = \mdseries, separator sign none, segmentation style = {solid, thmhlcol}
}{th}

\newtcbtheorem[number within = section]{corollary}{Corollary}{
  enhanced, breakable, colback = corbgcol!25,
  frame hidden, boxrule = 0sp, borderline west = {2pt}{0pt}{corhlcol},
  sharp corners, detach title, before upper = \tcbtitle\par\smallskip,
  coltitle = corhlcol, fonttitle = \bfseries,
  description font = \mdseries, separator sign none, segmentation style = {solid, corhlcol}
}{cor}

\newtcbtheorem[number within = section]{proposition}{Proposition}{
  enhanced, breakable, colback = propbgcol!25,
  frame hidden, boxrule = 0sp, borderline west = {2pt}{0pt}{prophlcol},
  sharp corners, detach title, before upper = \tcbtitle\par\smallskip,
  coltitle = prophlcol, fonttitle = \bfseries,
  description font = \mdseries, separator sign none, segmentation style = {solid, prophlcol}
}{prop}

\newtcbtheorem[number within = section]{example}{Example}{
  enhanced, breakable, colback = exbgcol!25,
  frame hidden, boxrule = 0sp, borderline west = {2pt}{0pt}{exhlcol},
  sharp corners, detach title, before upper = \tcbtitle\par\smallskip,
  coltitle = exhlcol, fonttitle = \bfseries,
  description font = \mdseries, separator sign none, segmentation style = {solid, exhlcol}
}{prop}

\newtcbtheorem[number within = section]{definition}{Definition}{
  enhanced, breakable, colback = red!10,
  frame hidden, boxrule = 0sp, borderline west = {2pt}{0pt}{red!50!black},
  sharp corners, detach title, before upper = \tcbtitle\par\smallskip,
  coltitle = red!50!black, fonttitle = \bfseries,
  description font = \mdseries, separator sign none, segmentation style = {solid, exhlcol}
}{def}

\makeatletter
\newtcbtheorem{question}{Question}{enhanced,
    breakable,
    colback=white,
    colframe=qheadcol,
    attach boxed title to top left={yshift*=-\tcboxedtitleheight},
    fonttitle=\bfseries,
    title={#2},
    boxed title size=title,
    boxed title style={%
            sharp corners,
            rounded corners=northwest,
            colback=qheadcol,
            boxrule=0pt,
        },
    underlay boxed title={%
            \path[fill=tcbcolframe] (title.south west)--(title.south east)
            to[out=0, in=180] ([xshift=5mm]title.east)--
            (title.center-|frame.east)
            [rounded corners=\kvtcb@arc] |-
            (frame.north) -| cycle;
        },
    #1
}{def}
\makeatother


\begin{document}

\maketitle

\begin{proposition}{}{}
  For disjoint rectangles $Q_{1}$ and $Q_{2}$, if the integrals exist:
  \[ \int_{Q_{1} \cup Q_{2}}f = \int_{Q_{1}}f + \int_{Q_{2}}f. \]
  You must show it using partitions. (I don't know if I can assume existence of both/one implying the other's)
\end{proposition}
\begin{proof}
  We assume that $f \ge 0$ \textendash{} we already take linearity/negation summing at this point.
  Let $Q_{1}, Q_{2} \subset \RR[n]$ be disjoint rectangles, and let $P$ partition their union.
  Define
  \[ f_{1}(x) = \begin{cases}
    f(x) \qquad \text{ if } x \in Q_{1}\\
    0 \quad\qquad \text{ otherwise }
  \end{cases} \]
and likewise for $f_{2}$.
We note that $f = f_{1} + f_{2}$ where we \emph{care}, and so we can rewrite our goal as showing
\[ \int_{Q_{1} \cup Q_{2}} f_{1} + f_{2} = \int_{Q_{1}} (f_{1} + f_{2}) + \int_{Q_{2}} (f_{1} + f_{2}). \]
As they are defined however, we see $f_{1} = 0$ on $Q_{2}$ and vice versa, leading to
\[ \int\limits_{Q_{1} \cup Q_{2}} f_{1} + f_{2} = \int_{Q_{1}} f_{1} + \int_{Q_{2}} f_{2}. \]
Since these integrals exist, we know that for $P$ we have
\[ L(f_{1}, P) + L(f_{2}, P) \le L(f_{1} + f_{2}, P) \le \int\limits_{Q_{1} \cup Q_{2}} (f_{1} + f_{2}) \le U(f_{1} + f_{2}, P) \le U(f_{1}, P) + U(f_{2}, P). \]
The inside three terms come from the existence of $\int f$ over the union, and the outer inequalities come from the triangle inequality as for any rectangle $R$ derived from $P$:
\[ m_{R}(f_{1}) + m_{R}(f_{2}) \le m_{R}(f_{1} + f_{2}) \]
and
\[ M_{R}(f_{1}) + M_{R}(f_{2}) \le M_{R}(f_{1} + f_{2}), \]
which imply the same inequalities for the lower and upper sums respectively.
However, the integrals over just $Q_{1}$ and $Q_{2}$ also exist, and by virtue of the outermost inequality $L(f_{1}) + L(f_{2}) \le U(f_{1}) + U(f_{2})$ we know that
the sum of the integrals is squished inside of this inequality chain as well.
As the lower and upper sums converge for the summed integrals, everything converges and becomes equal which in fact does show
\[ \int\limits_{Q_{1} \cup Q_{2}} f = \int_{Q_{1}} f + \int_{Q_{2}} f \]
for disjoint $Q_{i}$.
\end{proof}


\end{document}
