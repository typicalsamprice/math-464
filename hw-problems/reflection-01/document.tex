% -*- compile-command: "latexmk -pdf document.tex" -*-
\documentclass{article}

/home/sam/Git/latex-template/template.tex
\togglefalse{paper}

%%%%%%%%%%%%%%%%%%%%%
%%% THEOREM BOXES %%%
%%%%%%%%%%%%%%%%%%%%%
\usepackage[most,many,breakable]{tcolorbox}
\tcbuselibrary{theorems,skins,hooks}

% Stolen from: https://tex.stackexchange.com/a/330460
\makeatletter
\renewenvironment{proof}[1][\proofname]{\par
  \pushQED{\qed}%
  \normalfont \topsep6\p@\@plus6\p@\relax
  \trivlist
  \item[%
    \hskip\labelsep
    \normalfont\bfseries % was \itshape
    #1%
    \@addpunct{.}% remove this if you don't want punctuation
  ]\ignorespaces
}{%
  \popQED\endtrivlist\@endpefalse
}
\let\qed\relax % avoid a warning
\DeclareRobustCommand{\qed}{%
  \ifmmode \mathqed
  \else
    \leavevmode\unskip\penalty\@M\hbox{}\nobreak\hspace{.5em minus .1em}% was \hfill
    \hbox{\qedsymbol}%
  \fi
}
\makeatother

\ifprinted%
  \colorlet{thmbgcol}{white}
  \colorlet{lembgcol}{white}
  \colorlet{corbgcol}{white}
  \colorlet{propbgcol}{white}
  \colorlet{exbgcol}{white}
  \colorlet{defbgcol}{white}

  \colorlet{qheadcol}{black!20!white}
  \colorlet{qheadtxtcol}{black!90}

  \colorlet{exhlcol}{black}
  \colorlet{prophlcol}{black}
  \colorlet{corhlcol}{black}
  \colorlet{thmhlcol}{black}
  \colorlet{lemhlcol}{black}
  \colorlet{defhlcol}{black}
\else
  \definecolor{thmbgcol}{HTML}{aec1f9}
  \definecolor{corbgcol}{HTML}{b599f7}
  \definecolor{propbgcol}{HTML}{c9f7aa}
  \definecolor{exbgcol}{HTML}{f7c479}
  \colorlet{defbgcol}{red!7}

  \definecolor{qheadcol}{HTML}{182959}
  \colorlet{qheadtxtcol}{white}

  \definecolor{exhlcol}{HTML}{604419}
  \definecolor{prophlcol}{HTML}{1e631a}
  \definecolor{corhlcol}{HTML}{2d1760}
  \definecolor{thmhlcol}{HTML}{142c72}
  \colorlet{defhlcol}{red!50!black}

  \definecolor{lembgcol}{HTML}{e998f2}
  \definecolor{lemhlcol}{HTML}{791684}
\fi

\newtcbtheorem[number within = section]{theorem}{Theorem}{
  enhanced, breakable, colback = thmbgcol!20,
  frame hidden, boxrule = 0sp, borderline west = {2pt}{0pt}{thmhlcol},
  sharp corners, detach title, before upper = \tcbtitle\par\smallskip,
  coltitle = thmhlcol, fonttitle = \bfseries,
  description font = \mdseries, separator sign none, segmentation style = {solid, thmhlcol}
}{th}

\newtcbtheorem[number within = section]{lemma}{Lemma}{
  enhanced, breakable, colback = lembgcol!20,
  frame hidden, boxrule = 0sp, borderline west = {2pt}{0pt}{lemhlcol},
  sharp corners, detach title, before upper = \tcbtitle\par\smallskip,
  coltitle = lemhlcol, fonttitle = \bfseries,
  description font = \mdseries, separator sign none, segmentation style = {solid, lemhlcol}
}{lem}

\newtcbtheorem[number within = section]{corollary}{Corollary}{
  enhanced, breakable, colback = corbgcol!20,
  frame hidden, boxrule = 0sp, borderline west = {2pt}{0pt}{corhlcol},
  sharp corners, detach title, before upper = \tcbtitle\par\smallskip,
  coltitle = corhlcol, fonttitle = \bfseries,
  description font = \mdseries, separator sign none, segmentation style = {solid, corhlcol}
}{cor}

\newtcbtheorem[number within = section]{proposition}{Proposition}{
  enhanced, breakable, colback = propbgcol!25,
  frame hidden, boxrule = 0sp, borderline west = {2pt}{0pt}{prophlcol},
  sharp corners, detach title, before upper = \tcbtitle\par\smallskip,
  coltitle = prophlcol, fonttitle = \bfseries,
  description font = \mdseries, separator sign none, segmentation style = {solid, prophlcol}
}{prop}

\newtcbtheorem[number within = section]{example}{Example}{
  enhanced, breakable, colback = exbgcol!20,
  frame hidden, boxrule = 0sp, borderline west = {2pt}{0pt}{exhlcol},
  sharp corners, detach title, before upper = \tcbtitle\par\smallskip,
  coltitle = exhlcol, fonttitle = \bfseries,
  description font = \mdseries, separator sign none, segmentation style = {solid, exhlcol}
}{prop}

\newtcbtheorem[number within = section]{definition}{Definition}{
  enhanced, breakable, colback = defbgcol,
  frame hidden, boxrule = 0sp, borderline west = {2pt}{0pt}{defhlcol},
  sharp corners, detach title, before upper = \tcbtitle\par\smallskip,
  coltitle = defhlcol, fonttitle = \bfseries,
  description font = \mdseries, separator sign none, segmentation style = {solid, exhlcol}
}{def}

\makeatletter
\newtcbtheorem{question}{Question}{enhanced,
    breakable,
    colback=white,
    colframe=qheadcol,
    attach boxed title to top left={yshift*=-\tcboxedtitleheight},
    fonttitle=\bfseries,
    coltitle=qheadtxtcol,
    title={#2},
    boxed title size=title,
    boxed title style={%
            sharp corners,
            rounded corners=northwest,
            colback=qheadcol,
            boxrule=0pt,
        },
    underlay boxed title={%
            \path[fill=tcbcolframe] (title.south west)--(title.south east)
            to[out=0, in=180] ([xshift=5mm]title.east)--
            (title.center-|frame.east)
            [rounded corners=\kvtcb@arc] |-
            (frame.north) -| cycle;
        },
    #1
}{}
\makeatother


\begin{document}

\begin{center}
  \textbf{Self Reflection 1} \qquad Sam Price
\end{center}

This week I primarily spent my time working through some of the first exercises in Munkres, focusing on the inner product things, although I did spend quite some time thinking about the first real ``assigned''/presentable problem we got.

Let me first start with what I \emph{don't} understand quite yet: While working through the metric-induced norm problem, I came across a slight issue.
I came up with the translation invariance and homogeneity requirements, and could prove that said metric induces the norm $\norm{\indet} = d(\indet, 0)$.
When going the other way though, I was getting stuck and naturally looked for any pointers on the internet.
I came across a Math SE post (I cite this in my writeup for the problem, it is question 2888192)
that shows that the backwards direction \emph{does not} hold. That is, there is a metric (what they referred to as the ``french railway metric'') that creates the same norm without keeping translation invariance.
Said metric essentially is Euclidean distance but only allowing travel along lines through the origin.
Wikipedia's article on metric spaces doesn't say anything except for the direction I proved. Is what I \emph{should} be
proving something more like:
\begin{center}
  \fbox{\parbox{\textwidth/2}{
    A metric induces the norm $\norm{x - y} := d(x, y)$ if and only if it is homogeneous and translation-invariant.
  }}
\end{center}
I am going to guess that my struggles come from not understanding the problem; this most likely extends to the norm $\mapsto$ inner product requirements and proof. I'll talk to you tomorrow (or today, if that's when this gets turned in) about this.
I already have ideas about the boxed-in statement and have shown that the railway metric
doesn't hold for the target norm, which is good.

Onto the positives, what \emph{do} I understand this week? I learned about dual spaces, which is pretty neat, although I wouldn't say I have any intuitive or ``quick'' understanding of it at present.
I now also understand how an inner product space is in some sense one of the ``strongest'' vector spaces
since it has an induced norm and metric included for ``free''.
The biggest sense of knowledge gain thus far is easily looking at some of the theorems/proofs/observations
for matrix shenanigans in Munkres. Before I take linear algebra this will probably be my best exposure to it.

My goal for the next week (or two) is to work on more of the ``official'' problems as usual,
as well as to try to solidify my understanding of matrices and some of the things I don't quite get about linear algebra.
There's a lot of room for improvement there naturally, and I have an inkling that it may become important for a month or two for me to be more familiar with some of these topics.
Speaking of, why isn't math 355 a prereq for 464?
Is it just because it's a topics course and is \emph{very} volatile in content?

Since our class, I also have cofactor expansion to learn about and a couple more exercises to get ahead of.
The ones I plan to show you (as of now) is the aforementioned metric $\mapsto$ norm $\mapsto$ inner product chain, and
the matrix equivalency one since those seem the most $\ldots$important?

\vspace{2cm}
Unrelated but I thought it was funny (and perhaps concerning?) Apparently I was mumbling in my sleep saying ``therefore'', ``thus'' and ``derivative'' last night. My roommate is mildly concerned but what can ya do sometimes.


\end{document}
