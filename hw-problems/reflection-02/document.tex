% -*- compile-command: "latexmk -pdf document.tex" -*-
\documentclass[12pt]{article}

/home/sam/Git/latex-template/template.tex
\togglefalse{paper}

%%%%%%%%%%%%%%%%%%%%%
%%% THEOREM BOXES %%%
%%%%%%%%%%%%%%%%%%%%%
\usepackage[most,many,breakable]{tcolorbox}
\tcbuselibrary{theorems,skins,hooks}

% Stolen from: https://tex.stackexchange.com/a/330460
\makeatletter
\renewenvironment{proof}[1][\proofname]{\par
  \pushQED{\qed}%
  \normalfont \topsep6\p@\@plus6\p@\relax
  \trivlist
  \item[%
    \hskip\labelsep
    \normalfont\bfseries % was \itshape
    #1%
    \@addpunct{.}% remove this if you don't want punctuation
  ]\ignorespaces
}{%
  \popQED\endtrivlist\@endpefalse
}
\let\qed\relax % avoid a warning
\DeclareRobustCommand{\qed}{%
  \ifmmode \mathqed
  \else
    \leavevmode\unskip\penalty\@M\hbox{}\nobreak\hspace{.5em minus .1em}% was \hfill
    \hbox{\qedsymbol}%
  \fi
}
\makeatother

\ifprinted%
  \colorlet{thmbgcol}{white}
  \colorlet{lembgcol}{white}
  \colorlet{corbgcol}{white}
  \colorlet{propbgcol}{white}
  \colorlet{exbgcol}{white}
  \colorlet{defbgcol}{white}

  \colorlet{qheadcol}{black!20!white}
  \colorlet{qheadtxtcol}{black!90}

  \colorlet{exhlcol}{black}
  \colorlet{prophlcol}{black}
  \colorlet{corhlcol}{black}
  \colorlet{thmhlcol}{black}
  \colorlet{lemhlcol}{black}
  \colorlet{defhlcol}{black}
\else
  \definecolor{thmbgcol}{HTML}{aec1f9}
  \definecolor{corbgcol}{HTML}{b599f7}
  \definecolor{propbgcol}{HTML}{c9f7aa}
  \definecolor{exbgcol}{HTML}{f7c479}
  \colorlet{defbgcol}{red!7}

  \definecolor{qheadcol}{HTML}{182959}
  \colorlet{qheadtxtcol}{white}

  \definecolor{exhlcol}{HTML}{604419}
  \definecolor{prophlcol}{HTML}{1e631a}
  \definecolor{corhlcol}{HTML}{2d1760}
  \definecolor{thmhlcol}{HTML}{142c72}
  \colorlet{defhlcol}{red!50!black}

  \definecolor{lembgcol}{HTML}{e998f2}
  \definecolor{lemhlcol}{HTML}{791684}
\fi

\newtcbtheorem[number within = section]{theorem}{Theorem}{
  enhanced, breakable, colback = thmbgcol!20,
  frame hidden, boxrule = 0sp, borderline west = {2pt}{0pt}{thmhlcol},
  sharp corners, detach title, before upper = \tcbtitle\par\smallskip,
  coltitle = thmhlcol, fonttitle = \bfseries,
  description font = \mdseries, separator sign none, segmentation style = {solid, thmhlcol}
}{th}

\newtcbtheorem[number within = section]{lemma}{Lemma}{
  enhanced, breakable, colback = lembgcol!20,
  frame hidden, boxrule = 0sp, borderline west = {2pt}{0pt}{lemhlcol},
  sharp corners, detach title, before upper = \tcbtitle\par\smallskip,
  coltitle = lemhlcol, fonttitle = \bfseries,
  description font = \mdseries, separator sign none, segmentation style = {solid, lemhlcol}
}{lem}

\newtcbtheorem[number within = section]{corollary}{Corollary}{
  enhanced, breakable, colback = corbgcol!20,
  frame hidden, boxrule = 0sp, borderline west = {2pt}{0pt}{corhlcol},
  sharp corners, detach title, before upper = \tcbtitle\par\smallskip,
  coltitle = corhlcol, fonttitle = \bfseries,
  description font = \mdseries, separator sign none, segmentation style = {solid, corhlcol}
}{cor}

\newtcbtheorem[number within = section]{proposition}{Proposition}{
  enhanced, breakable, colback = propbgcol!25,
  frame hidden, boxrule = 0sp, borderline west = {2pt}{0pt}{prophlcol},
  sharp corners, detach title, before upper = \tcbtitle\par\smallskip,
  coltitle = prophlcol, fonttitle = \bfseries,
  description font = \mdseries, separator sign none, segmentation style = {solid, prophlcol}
}{prop}

\newtcbtheorem[number within = section]{example}{Example}{
  enhanced, breakable, colback = exbgcol!20,
  frame hidden, boxrule = 0sp, borderline west = {2pt}{0pt}{exhlcol},
  sharp corners, detach title, before upper = \tcbtitle\par\smallskip,
  coltitle = exhlcol, fonttitle = \bfseries,
  description font = \mdseries, separator sign none, segmentation style = {solid, exhlcol}
}{prop}

\newtcbtheorem[number within = section]{definition}{Definition}{
  enhanced, breakable, colback = defbgcol,
  frame hidden, boxrule = 0sp, borderline west = {2pt}{0pt}{defhlcol},
  sharp corners, detach title, before upper = \tcbtitle\par\smallskip,
  coltitle = defhlcol, fonttitle = \bfseries,
  description font = \mdseries, separator sign none, segmentation style = {solid, exhlcol}
}{def}

\makeatletter
\newtcbtheorem{question}{Question}{enhanced,
    breakable,
    colback=white,
    colframe=qheadcol,
    attach boxed title to top left={yshift*=-\tcboxedtitleheight},
    fonttitle=\bfseries,
    coltitle=qheadtxtcol,
    title={#2},
    boxed title size=title,
    boxed title style={%
            sharp corners,
            rounded corners=northwest,
            colback=qheadcol,
            boxrule=0pt,
        },
    underlay boxed title={%
            \path[fill=tcbcolframe] (title.south west)--(title.south east)
            to[out=0, in=180] ([xshift=5mm]title.east)--
            (title.center-|frame.east)
            [rounded corners=\kvtcb@arc] |-
            (frame.north) -| cycle;
        },
    #1
}{}
\makeatother


\author{Sam Price}
\date{}
\title{Reflection Two: Overanalyzed Goo}

\begin{document}

\maketitle

This week's work has been pretty interesting.
So far, my best moments in this class have come from \emph{understanding new things} for the most part.
Once we started working in $\RR[n]$ and thinking about differentiation more, I've gotten a bit better at really understanding why things are the way they are.
The issue with this is realizing I have to do more exercises from the book; everything is divinely interpreted at present and I think doing more with these objects
will give me a better intuition for how they operate. In particular, I think the transition to a derivative being a \emph{matrix} was a pretty big shock at first but has come
to mean more now that I've realized why ``the derivative'' could be thought of as a $1 \times 1$ matrix before.

The biggest mindblock this week while working on problems comes in two parts: the first
was cleared up today after my use of $\sup$ in $\RR[n]$ was declared okay.
I didn't know how much I could use on faith (all the way to MCT?) but now with the power
of the supremum I am confident I can tackle completeness without any major issues.
As I said earlier, showing MCT and Bolzano-Weierstrass and using the fact Cauchy sequences are
inherently bounded will give me a completeness proof.

The second issue I am having with an exercise is the determinancy equivalency.
I'm guessing Spence $\implies$ Munkres isn't as difficult, since we can easily leverage
the fact that $\det AB = \det A \det B$. I'm wondering if the optimal solution path to this one
is showing that $\det EA = \det E \det A$ for an elementary matrix.
This feels like a similar argument I remember reading later in the book,
so I'm sure something like that can be possible and make life easier.
Going back the other way would involve something similar, and is manageable if so.

This week I will most likely not have too much time to dedicate to this (or any math)
and so will have to optimize for the classes that have time constraints.
After this weekend though, I want to grind out a decent amount of exercises/convincing myself of things
in the textbook. Particularly focused on the topology aspects in part because they are very fun for me
(who would have guessed) and also because using the sup norm more often to understand its use
cases and strengths feels like something I need to build up.

The rest of my time will be spent on differentiation, since that's what we have only just covered to
(some of?) its full extent so far. Once I really understood my own algebra mistakes and mishaps
today, it more or less fell into place. This really got you to show your insanity with partials though,
because I don't think I'll ever be able to look at
\[ D_{i}D_{j}f(a) \in \Lin(\RR[m], \Lin(\RR[m], \RR[n])) \iso \Lin(\RR[m] \times \RR[m], \RR[n]) \]
and think whoever invented this is normal. Better than saying $\Hom$ I suppose, but still feels a tad silly.
On a more serious note however, I think this is the first really complicatedly new thing that isn't (just)
linear algebra that I went from almost no understanding to at least having the ego to say I actually
do get the concepts. It was also nice to use actual notation/concepts like the $C^{r}$ classes
instead of doing the Calc 3 tactic of ``if the function is sufficiently nice.''
Granted, actually having the concrete assumption makes the proof (rigorously) possible so at a
certain point I shouldn't expect the lack of precise descriptions.

An interesting thought that came to mind tonight: can a function $f \from \bbQ \to \bbR$ be continuous?
Intuitively no, since $\bbQ$ isn't complete. Once the intuition faded though I am questioning whether or not
any part of the definition of continuity actually \emph{requires} that. $\bbQ$ is dense after all, and so
no disproof immediately comes to mind. Food for my brain to chew on sometime.
\end{document}
