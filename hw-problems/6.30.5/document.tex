% -*- compile-command: "latexmk -pdf document.tex" -*-
\documentclass{article}

\author{Sam Price}
\title{Chapter 6, Section 30, Ex. 5}

\newif\ifprinted%
\printedtrue%
% Most taken from: https://github.com/SeniorMars/dotfiles/blob/main/latex_template/preamble.tex
\usepackage{geometry}

\usepackage[english]{babel}
\usepackage[T1]{fontenc}
\usepackage[utf8]{inputenc}
\usepackage{palatino}

% Use the command \doublespacing if needed
\usepackage{setspace}

\geometry{a4paper, margin=1in}

\usepackage{mathtools}
\usepackage{amssymb,amsfonts,amsthm,amsmath}

\usepackage{xfrac}
\usepackage[makeroom]{cancel}

% Use \begin{enumerate}[start=x,label={Q\arabic*)}] for example
\usepackage{enumitem}

\usepackage{xcolor}

\usepackage{nameref}

% Important options
% in envs, use options like [baseline=x] to center on row x (or special opts t/[c]/b without 'baseline')
\usepackage{nicematrix}
\NiceMatrixOptions{cell-space-limits = 1pt}

\usepackage{booktabs}

\usepackage{tikz}
\usepackage{tikz-cd}
\usepackage{tikzsymbols}

\usepackage{pdfpages}

\usepackage[most,many,breakable]{tcolorbox}

\setlength{\parindent}{1cm}

\DeclarePairedDelimiter{\abs}{\lvert}{\rvert}

%%%%%%%%%%%%%%%%%%%%%
%%% THEOREM BOXES %%%
%%%%%%%%%%%%%%%%%%%%%
\definecolor{thmbgcol}{HTML}{aec1f9}
\definecolor{thmhlcol}{HTML}{142c72}
\definecolor{corbgcol}{HTML}{b599f7}
\definecolor{corhlcol}{HTML}{2d1760}
\definecolor{propbgcol}{HTML}{c9f7aa}
\definecolor{prophlcol}{HTML}{1e631a}
\definecolor{exbgcol}{HTML}{f7c479}
\definecolor{exhlcol}{HTML}{604419}

\definecolor{qheadcol}{HTML}{182959}

\tcbuselibrary{theorems,skins,hooks}
\newtcbtheorem[number within = section]{theorem}{Theorem}{
  enhanced, breakable, colback = thmbgcol!25,
  frame hidden, boxrule = 0sp, borderline west = {2pt}{0pt}{thmhlcol},
  sharp corners, detach title, before upper = \tcbtitle\par\smallskip,
  coltitle = thmhlcol, fonttitle = \bfseries,
  description font = \mdseries, separator sign none, segmentation style = {solid, thmhlcol}
}{th}

\newtcbtheorem[number within = section]{corollary}{Corollary}{
  enhanced, breakable, colback = corbgcol!25,
  frame hidden, boxrule = 0sp, borderline west = {2pt}{0pt}{corhlcol},
  sharp corners, detach title, before upper = \tcbtitle\par\smallskip,
  coltitle = corhlcol, fonttitle = \bfseries,
  description font = \mdseries, separator sign none, segmentation style = {solid, corhlcol}
}{cor}

\newtcbtheorem[number within = section]{proposition}{Proposition}{
  enhanced, breakable, colback = propbgcol!25,
  frame hidden, boxrule = 0sp, borderline west = {2pt}{0pt}{prophlcol},
  sharp corners, detach title, before upper = \tcbtitle\par\smallskip,
  coltitle = prophlcol, fonttitle = \bfseries,
  description font = \mdseries, separator sign none, segmentation style = {solid, prophlcol}
}{prop}

\newtcbtheorem[number within = section]{example}{Example}{
  enhanced, breakable, colback = exbgcol!25,
  frame hidden, boxrule = 0sp, borderline west = {2pt}{0pt}{exhlcol},
  sharp corners, detach title, before upper = \tcbtitle\par\smallskip,
  coltitle = exhlcol, fonttitle = \bfseries,
  description font = \mdseries, separator sign none, segmentation style = {solid, exhlcol}
}{prop}

\newtcbtheorem[number within = section]{definition}{Definition}{
  enhanced, breakable, colback = red!10,
  frame hidden, boxrule = 0sp, borderline west = {2pt}{0pt}{red!50!black},
  sharp corners, detach title, before upper = \tcbtitle\par\smallskip,
  coltitle = red!50!black, fonttitle = \bfseries,
  description font = \mdseries, separator sign none, segmentation style = {solid, exhlcol}
}{def}

\makeatletter
\newtcbtheorem{question}{Question}{enhanced,
    breakable,
    colback=white,
    colframe=qheadcol,
    attach boxed title to top left={yshift*=-\tcboxedtitleheight},
    fonttitle=\bfseries,
    title={#2},
    boxed title size=title,
    boxed title style={%
            sharp corners,
            rounded corners=northwest,
            colback=qheadcol,
            boxrule=0pt,
        },
    underlay boxed title={%
            \path[fill=tcbcolframe] (title.south west)--(title.south east)
            to[out=0, in=180] ([xshift=5mm]title.east)--
            (title.center-|frame.east)
            [rounded corners=\kvtcb@arc] |-
            (frame.north) -| cycle;
        },
    #1
}{def}
\makeatother


\begin{document}

\maketitle

\begin{theorem}{}{}
  Let $A = \RR[2] \setminus \set{0}$ and
  \[ \omega = \frac{-y\, dx + x \, dy}{x^{2} + y^{2}}. \]
  Show $\omega$ is closed ($d\omega = 0$) but not exactly ($\nexists f, \omega = df$) on $A$.
\end{theorem}

\begin{proof}

\begin{enumerate}[start=1,label={(\alph*)}]
  \item Show $\omega$ is closed.

        We can purely calculate that
        \begin{align*}
          d\omega &= d\pars*{\frac{-y}{x^{2} + y^{2}} dx + \frac{x}{x^{2} + y^{2}} dy}\\
          &= \frac{y^{2} - x^{2}}{x^{2} + y^{2}} dx \wedge dy
            + \frac{y^{2} - x^{2}}{x^{2} + y^{2}} dy \wedge dx.
        \end{align*}
        By flipping signs so we can add these forms, it is clear the sum is 0 and
        so $\omega$ is closed.

  \item Let $B = A \setminus ((0, \infty) \times \set{0})$.
        Show that for $(x, y) \in B$ there is a unique $t \in (0, 2\pi)$ so that
        \[ x = \sqrt{x^{2} + y^{2}} \cos t \]
        and \[ y = \sqrt{x^{2} + y^{2}} \sin t \]

        During the midterm this was shown to be true, and that a piecewise $\arctan$ function
        is appropriate. This function is called $\phi(x, y)$ and gives the correct $t$.
        This gives the correct value as it translates between polar and Cartesian coordinates
        on $B$.

  \item Show $\phi$ is smooth.

        Since this is the arctangent function with holes filled in when $x = 0$ and
        some offsets, we only need to check that (i) the filled-in points are in fact the limits
        and (ii) the offsets don't break smoothness.

        For (i) it is clear to see when $x = 0$ that crossing the positive $y$-axis
        should be going towards exactly $\pi/2$ from either side. The same is true for the
        negative $y$-axis giving $3\pi/2$ as the limit on both sides.

        To affirm (ii), we see that $\phi(x, 0)$ should be $\pi$ as expected for $x < 0$
        from above and below.
        Since we don't include the positive $x$-axis in $B$, we will not look at it here.

  \item Show $\omega = d\phi$ in $B$.

        Since $\phi = \arctan$ up to a constant, we can see that
        \[ d\phi = \frac{\partial}{\partial x}\arctan(y/x)\, dx + \frac{\partial}{\partial y}\arctan(y/x) \, dy \]
        and so
        \[ d\phi = \frac{-y \, dx}{x^{2} + y^{2}} + \frac{x \, dy}{x^{2} + y^{2}} \]
        which is exactly $\omega$.

  \item Show if $g \in \Omega^{0}(B)$ is closed, then $g$ is constant.

        Let $a = (-1, 0)$. By the mean value theorem, we see for all $x \in B$ distinct from $a$
        there is some $y \in B$ so that
        \[ Dg(y) = \frac{Dg(x) - Dg(a)}{\norm{x - a}}. \]
        Since $g$ is closed, $D_{1}g = D_{2}g = 0$ and so $g$ must be constant.

  \item Show $\omega$ is not exact in $A$.

        Suppose $\omega = df$ for some $f \in \Omega^{0}(A)$.
        Then, $f|_{B} = g \in \Omega^{0}(B)$ and $g = \phi$ on $B$.
        If $g$ agrees with $\phi$ on $B$ however, it cannot be re-extended back to $A$ without
        being discontinuous along the positive $x$-axis. This is because there would necessarily
        be a jump discontinuity along this part of the axis and so $\omega \ne df$.
\end{enumerate}

\end{proof}


\end{document}
