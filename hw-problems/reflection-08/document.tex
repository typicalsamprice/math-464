% -*- compile-command: "latexmk -pdf document.tex" -*-
\documentclass[12pt]{article}

\author{Sam Price}
\title{Reflection Eight}

\newif\ifprinted%
%\printedtrue%
% Most taken from: https://github.com/SeniorMars/dotfiles/blob/main/latex_template/preamble.tex
\usepackage{geometry}

\usepackage[english]{babel}
\usepackage[T1]{fontenc}
\usepackage[utf8]{inputenc}
\usepackage{palatino}

% Use the command \doublespacing if needed
\usepackage{setspace}

\geometry{a4paper, margin=1in}

\usepackage{mathtools}
\usepackage{amssymb,amsfonts,amsthm,amsmath}

\usepackage{xfrac}
\usepackage[makeroom]{cancel}

% Use \begin{enumerate}[start=x,label={Q\arabic*)}] for example
\usepackage{enumitem}

\usepackage{xcolor}

\usepackage{nameref}

% Important options
% in envs, use options like [baseline=x] to center on row x (or special opts t/[c]/b without 'baseline')
\usepackage{nicematrix}
\NiceMatrixOptions{cell-space-limits = 1pt}

\usepackage{booktabs}

\usepackage{tikz}
\usepackage{tikz-cd}
\usepackage{tikzsymbols}

\usepackage{pdfpages}

\usepackage[most,many,breakable]{tcolorbox}

\setlength{\parindent}{1cm}

\DeclarePairedDelimiter{\abs}{\lvert}{\rvert}

%%%%%%%%%%%%%%%%%%%%%
%%% THEOREM BOXES %%%
%%%%%%%%%%%%%%%%%%%%%
\definecolor{thmbgcol}{HTML}{aec1f9}
\definecolor{thmhlcol}{HTML}{142c72}
\definecolor{corbgcol}{HTML}{b599f7}
\definecolor{corhlcol}{HTML}{2d1760}
\definecolor{propbgcol}{HTML}{c9f7aa}
\definecolor{prophlcol}{HTML}{1e631a}
\definecolor{exbgcol}{HTML}{f7c479}
\definecolor{exhlcol}{HTML}{604419}

\definecolor{qheadcol}{HTML}{182959}

\tcbuselibrary{theorems,skins,hooks}
\newtcbtheorem[number within = section]{theorem}{Theorem}{
  enhanced, breakable, colback = thmbgcol!25,
  frame hidden, boxrule = 0sp, borderline west = {2pt}{0pt}{thmhlcol},
  sharp corners, detach title, before upper = \tcbtitle\par\smallskip,
  coltitle = thmhlcol, fonttitle = \bfseries,
  description font = \mdseries, separator sign none, segmentation style = {solid, thmhlcol}
}{th}

\newtcbtheorem[number within = section]{corollary}{Corollary}{
  enhanced, breakable, colback = corbgcol!25,
  frame hidden, boxrule = 0sp, borderline west = {2pt}{0pt}{corhlcol},
  sharp corners, detach title, before upper = \tcbtitle\par\smallskip,
  coltitle = corhlcol, fonttitle = \bfseries,
  description font = \mdseries, separator sign none, segmentation style = {solid, corhlcol}
}{cor}

\newtcbtheorem[number within = section]{proposition}{Proposition}{
  enhanced, breakable, colback = propbgcol!25,
  frame hidden, boxrule = 0sp, borderline west = {2pt}{0pt}{prophlcol},
  sharp corners, detach title, before upper = \tcbtitle\par\smallskip,
  coltitle = prophlcol, fonttitle = \bfseries,
  description font = \mdseries, separator sign none, segmentation style = {solid, prophlcol}
}{prop}

\newtcbtheorem[number within = section]{example}{Example}{
  enhanced, breakable, colback = exbgcol!25,
  frame hidden, boxrule = 0sp, borderline west = {2pt}{0pt}{exhlcol},
  sharp corners, detach title, before upper = \tcbtitle\par\smallskip,
  coltitle = exhlcol, fonttitle = \bfseries,
  description font = \mdseries, separator sign none, segmentation style = {solid, exhlcol}
}{prop}

\newtcbtheorem[number within = section]{definition}{Definition}{
  enhanced, breakable, colback = red!10,
  frame hidden, boxrule = 0sp, borderline west = {2pt}{0pt}{red!50!black},
  sharp corners, detach title, before upper = \tcbtitle\par\smallskip,
  coltitle = red!50!black, fonttitle = \bfseries,
  description font = \mdseries, separator sign none, segmentation style = {solid, exhlcol}
}{def}

\makeatletter
\newtcbtheorem{question}{Question}{enhanced,
    breakable,
    colback=white,
    colframe=qheadcol,
    attach boxed title to top left={yshift*=-\tcboxedtitleheight},
    fonttitle=\bfseries,
    title={#2},
    boxed title size=title,
    boxed title style={%
            sharp corners,
            rounded corners=northwest,
            colback=qheadcol,
            boxrule=0pt,
        },
    underlay boxed title={%
            \path[fill=tcbcolframe] (title.south west)--(title.south east)
            to[out=0, in=180] ([xshift=5mm]title.east)--
            (title.center-|frame.east)
            [rounded corners=\kvtcb@arc] |-
            (frame.north) -| cycle;
        },
    #1
}{def}
\makeatother


\begin{document}

\maketitle

This week has been incredibly stressful in terms of math, and I don't expect that to change any time soon.
Next semester should be a far lesser load, looking for roughly 15 credits instead of 20, with 2 math + 2 CS + PWP 372. Never again.

Onto math! In terms of positives, this week has been kind to me.
I'm starting to grasp some concepts around manifolds, and it has been interesting to see some of the relations coming together with algebraic geometry.
Iain and I just got done working on a couple things (it is 9 pm, roughly) and that had some fruitful results as well.
We worked on the problem involving the manifold defined by
\[ a(t) = \begin{pmatrix} t + t^{2} \\ t^{2} \\ t^{3} \end{pmatrix}. \]
We got as far as showing $\alpha$ itself is a diffeomorphism from $\RR \to \RR[3]$, but we didn't understand my notes of what the problem (2) and (3) parts were requesting.

The other problem we got halfway through was showing $\GL_{n}$ and $\SL_{n}$ are manifolds.
We showed that since $\GL_{n} \subset \RR[n^{2}]$ is open ($\GL_{n} = \det\inv(\RR \setminus \set{0})$) and $\RR[n^{2}]$ is a manifold,
then $\GL_{n}$ must also by leveraging the implicit manifold definition and restrictions.
Then, we showed that $f_{A}\from \GL_{n} \to \GL_{n}$ defined by $B \mapsto AB$ is smooth since its derivative is $\pars{\det B}^{2} > 0$.
As $A$ is generic over $\GL_{n}$, this is actually equivalent to showing the binary operation is smooth as well.
Further, taking the inverses were smooth since that can be computed as a series of multiplications, and so $\GL_{n}$ is a real Lie group!
With the $\SL_{n}$ part we decided we couldn't really get far enough yet, and so will revisit after a couple more classes most likely.

Struggles this week were not limited to MATH 464.
I've spent both a decent amount of time and simultaneously not enough on exercises in the textbook.
As we get further into manifolds, I really do just need to bite the bullet and buckle down for a weekend or two.
In terms of fundamental concepts, I don't fully get why the implicit manifold part is useful.
How can a $C^{r}$ function have a preimage of 0 that is $M \cap W$? From what I could tell $W$ had to be open, so how is $F\inv(0)$ equal?
Is this an artifact of $M$ being (probably?) closed in its larger space and it requires a transformation to understand directly?

I'm going to also try and make as many corrections/additions to my presentation tomorrow morning and during my CS class.
No promises I get \emph{everything} done, but I will do as much as is feasible.

In terms of my midterm, I'm very curious as to what the generalized case of (2) looks like.
I more or less made all of that up without understanding it, just making it sound like the previous steps.
Looking back a couple mistakes are a tad silly, as per usual, but I'll settle at times for knowing what I thought I meant was correct even
if I absolutely bombed transferring that idea to paper. Not ideal, but \emph{sometimes} I have to make mistakes I \emph{suppose}.

In other math classes, I did have a bit of fun.
In abstract, I decided instead of doing some inspection/casework I would come up with a lemma and prove it, then give one-word answers
to each part. Luckily, what I hypothesized was true and inspired by a previous exercise (product of cyclic groups is cyclic iff their orders are coprime.)
This was really just because it seems very funny to overcomplicate something initially and then rest on my laurels for the bit.

Also, I lied about the programmer-turned-farmer today, Dylan Araps. I said his largest project was neofetch; turns out neofetch ``only'' has 22k stars, and his
project titled ``pure bash bible'', which is just a collection of bash snippets/ideas (hence ``bible'' - this is a series across many creators)
has over 36k stars. Not even CLOSE to the 100k I made up on the spot, but still incredibly impressive.
His tool pywal (generates colorschemes for \emph{everything} based on wallpaper colors) only has about eight thousand,
which is still really cool for an independent open-sourcer admittedly.

\end{document}
