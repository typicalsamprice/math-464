% -*- compile-command: "latexmk -pdf document.tex" -*-
\documentclass{article}

\author{Sam Price}
\date{Last Updated \today}
\title{Advanced Calculus Notes}

\newif\ifprinted%
%\printedtrue%
% Most taken from: https://github.com/SeniorMars/dotfiles/blob/main/latex_template/preamble.tex
\usepackage{geometry}

\usepackage[english]{babel}
\usepackage[T1]{fontenc}
\usepackage[utf8]{inputenc}
\usepackage{palatino}

% Use the command \doublespacing if needed
\usepackage{setspace}

\geometry{a4paper, margin=1in}

\usepackage{mathtools}
\usepackage{amssymb,amsfonts,amsthm,amsmath}

\usepackage{xfrac}
\usepackage[makeroom]{cancel}

% Use \begin{enumerate}[start=x,label={Q\arabic*)}] for example
\usepackage{enumitem}

\usepackage{xcolor}

\usepackage{nameref}

% Important options
% in envs, use options like [baseline=x] to center on row x (or special opts t/[c]/b without 'baseline')
\usepackage{nicematrix}
\NiceMatrixOptions{cell-space-limits = 1pt}

\usepackage{booktabs}

\usepackage{tikz}
\usepackage{tikz-cd}
\usepackage{tikzsymbols}

\usepackage{pdfpages}

\usepackage[most,many,breakable]{tcolorbox}

\setlength{\parindent}{1cm}

\DeclarePairedDelimiter{\abs}{\lvert}{\rvert}

%%%%%%%%%%%%%%%%%%%%%
%%% THEOREM BOXES %%%
%%%%%%%%%%%%%%%%%%%%%
\definecolor{thmbgcol}{HTML}{aec1f9}
\definecolor{thmhlcol}{HTML}{142c72}
\definecolor{corbgcol}{HTML}{b599f7}
\definecolor{corhlcol}{HTML}{2d1760}
\definecolor{propbgcol}{HTML}{c9f7aa}
\definecolor{prophlcol}{HTML}{1e631a}
\definecolor{exbgcol}{HTML}{f7c479}
\definecolor{exhlcol}{HTML}{604419}

\definecolor{qheadcol}{HTML}{182959}

\tcbuselibrary{theorems,skins,hooks}
\newtcbtheorem[number within = section]{theorem}{Theorem}{
  enhanced, breakable, colback = thmbgcol!25,
  frame hidden, boxrule = 0sp, borderline west = {2pt}{0pt}{thmhlcol},
  sharp corners, detach title, before upper = \tcbtitle\par\smallskip,
  coltitle = thmhlcol, fonttitle = \bfseries,
  description font = \mdseries, separator sign none, segmentation style = {solid, thmhlcol}
}{th}

\newtcbtheorem[number within = section]{corollary}{Corollary}{
  enhanced, breakable, colback = corbgcol!25,
  frame hidden, boxrule = 0sp, borderline west = {2pt}{0pt}{corhlcol},
  sharp corners, detach title, before upper = \tcbtitle\par\smallskip,
  coltitle = corhlcol, fonttitle = \bfseries,
  description font = \mdseries, separator sign none, segmentation style = {solid, corhlcol}
}{cor}

\newtcbtheorem[number within = section]{proposition}{Proposition}{
  enhanced, breakable, colback = propbgcol!25,
  frame hidden, boxrule = 0sp, borderline west = {2pt}{0pt}{prophlcol},
  sharp corners, detach title, before upper = \tcbtitle\par\smallskip,
  coltitle = prophlcol, fonttitle = \bfseries,
  description font = \mdseries, separator sign none, segmentation style = {solid, prophlcol}
}{prop}

\newtcbtheorem[number within = section]{example}{Example}{
  enhanced, breakable, colback = exbgcol!25,
  frame hidden, boxrule = 0sp, borderline west = {2pt}{0pt}{exhlcol},
  sharp corners, detach title, before upper = \tcbtitle\par\smallskip,
  coltitle = exhlcol, fonttitle = \bfseries,
  description font = \mdseries, separator sign none, segmentation style = {solid, exhlcol}
}{prop}

\newtcbtheorem[number within = section]{definition}{Definition}{
  enhanced, breakable, colback = red!10,
  frame hidden, boxrule = 0sp, borderline west = {2pt}{0pt}{red!50!black},
  sharp corners, detach title, before upper = \tcbtitle\par\smallskip,
  coltitle = red!50!black, fonttitle = \bfseries,
  description font = \mdseries, separator sign none, segmentation style = {solid, exhlcol}
}{def}

\makeatletter
\newtcbtheorem{question}{Question}{enhanced,
    breakable,
    colback=white,
    colframe=qheadcol,
    attach boxed title to top left={yshift*=-\tcboxedtitleheight},
    fonttitle=\bfseries,
    title={#2},
    boxed title size=title,
    boxed title style={%
            sharp corners,
            rounded corners=northwest,
            colback=qheadcol,
            boxrule=0pt,
        },
    underlay boxed title={%
            \path[fill=tcbcolframe] (title.south west)--(title.south east)
            to[out=0, in=180] ([xshift=5mm]title.east)--
            (title.center-|frame.east)
            [rounded corners=\kvtcb@arc] |-
            (frame.north) -| cycle;
        },
    #1
}{def}
\makeatother


\begin{document}

\maketitle

\section{Linear Algebra}

\subsection{Determinants}

\section{Derivatives}

\subsection{Inverse Function Theorem}

This is my capstone theorem.

\subsection{Implicit Function Theorem}

\section{Integrals}

\begin{definition}{Rectifiable Sets}{rectifiable}
  A set $S \subset \RR[n]$ is \emph{rectifiable} if and only if
  $S$ is bounded and $\Bd S$ is measure zero.
\end{definition}


\subsection{Extended Integrals}

\subsection{Change of Variables}

\subsubsection{Isometries}

An isometry is a transformation $h \from \RR[n] \to \RR[n]$ such that distances are preserved.
Generally doesn't necessarily hold, but this will be true over $n$-dimensional vector spaces over $\RR$.
Formally, this means that for a metric $d$ over $\RR[n]$, there is
\[ d(x, y) = d(h(x), h(y)). \]

An \emph{orthogonal} transformation is a transformation such that angles (dot products) are preserved in $\RR[n]$.
These are the linear transformations whose defining matrix has columns that are an \emph{orthonormal basis}.

\begin{theorem}{}{isometry-is-affine}
  For $h \from \RR[n] \to \RR[n]$, $h$ is an isometry if and only if $h(x) = Ax + p$
  for some fixed $A \in \Orth(n)$ and $p \in \RR[n]$.
\end{theorem}

Furthermore, these seem quite related and so:

\begin{theorem}{}{iso-zero-fixed-iff-ortho-and-dp-preserved}
  If $h \from \RR[n] \to \RR[n]$ and $h(0) = 0$, then $h$ is an isometry if and only if
  \begin{enumerate}[start=1,label={\arabic*\rparen}]
    \item $h$ preserves dot products. That is, $\angles{x, y} = \angles{h(x), h(y)}$, and
    \item $h$ is an orthogonal transformation.
  \end{enumerate}
\end{theorem}

\begin{lemma}{}{}
  Let $A \subset \RR[n]$ be open and $f \from A \to \RR$ be continuous.
  If $f$ vanishes outside a compact subset $C \subset A$, then
  \[ \int_{A}f = \int_{C}f. \]
\end{lemma}
\begin{proof}
  Since $C$ is bounded and $f\mid_{C}$ is continuous, then $\int_{C}f$ exists.
  Choose $\set{C_{i}}$ compact, rectifiable sets that cover $A$ so that $C_{i} \subset \Int C_{i + 1}$.
  So, there is some $N$ so that $C \subset C_{N}$.
  Thus,
  \[ \int_{C}f = \int_{C_{n}}f \quad\forall n \ge N. \]
  Because $\pars*{\int_{C_{n}}f} \to \int_{C}f$ and $(C_{i}) \to A$, we see
  \[ \int_{C}f = \int_{A}f. \]
\end{proof}

\begin{theorem}{}{}
  Let $A \subset \RR[n]$ be open, $f \from A \to \RR$ continuous and $\set{\phi_{i}}$ be a partition of unity.
  Then $\int_{A}f$ exists if and only if
  \[ \sum_{i = 1}^{\infty}\int_{A}\phi_{i}\abs{f}, \]
  in which case
  \[ \int_{A}f = \sum_{i = 1}^{\infty}\int_{A}\phi_{i}f. \]
\end{theorem}
\begin{proof}
  Assume $f = \abs{f}$. Let $D \subset A$ be compact and rectifiable.
  Then, there is some $M \in \NN$ so that for $m > M$, $\phi_{m} = 0$  on $D$.
  In fact, $f_{D}(x) = \pars{\sum_{i = 1}^{M}\phi_{i}f}(x)$.
  Thus,
  \begin{align*}
    \int_{D}f &= \sum_{i = 1}^{M}\pars*{\int_{D}\phi_{i}f}\\
    &\le \sum_{i = 1}^{M}\pars*{\int_{D \cup \Support \phi_{i}} \phi_{i}f}\\
    &= \sum_{i = 1}^{M}\pars*{\int_{A}\phi_{i}f}\\
    &\le \sum_{i = 1}^{\infty}\pars*{\int_{A}\phi_{i}f}.
  \end{align*}

  Since the last inequality is an upper bound for $\int_{D}f$, we know that if the former exists then
  \[ \int_{A}f = \sum_{i = 1}^{\infty}\pars*{\int_{A}\phi_{i}f}. \]

  Now suppose $\int_{A}f$ exists. Given $N \in \NN$, set $D = \bigcup_{i \le N} \Support \phi_{i}$.
  For $i \le N$, there is
  \[ \int_{A}\phi_{i}f = \int_{D}\phi_{i}f \]
  since each $\phi$ vanishes outside $D$ on $A$. Then, we have
  \begin{align*}
    \sum_{i = 1}^{N}\pars*{\int_{A}\phi_{i}f} &= \sum_{i = 1}^{N}\pars*{\int_{D}\phi_{i}f}\\
    &= \int_{D}\sum_{i = 1}^{N}\phi_{i}f\\
    &= \int_{D}f.
  \end{align*}
  Since the integral on $D$ must be less than (or equal to) the extended integral on $A$,
  we use the fact that $f = \abs{f}$ and so the first sum must converge in $\RR$ as it is monotone.
\end{proof}

\section{Manifolds}

We want to, in some sense, ``do calculus'' on subsets of $\RR[n]$.

\begin{definition}{Regular Parametrized Curve}{reg-param-curve}
  A regular parametrized curve is a map $\alpha \from I \to \RR[n]$ where
  $I \subset \RR$ is open, $\alpha$ is smooth and $\D\alpha \ne 0$ for any $t \in I$.
\end{definition}


\begin{definition}{Manifold Prototype I}{manifold-proto-i}
  A 1-manifold is a set $X \subset \RR[n]$ which is \underline{locally}
  an injective regular parametrized curve.
\end{definition}

This means for a manifold $X$, locally for all $p \in X$,
there is some open $U \subset \RR[n]$ so that
\[ \alpha \from I \to U \cap X \]
is such an injective regular parametrized curve (IRPC).

%%%%%%%%%%%

\begin{example}{Exercise}{}
  Prove that the union of the $x$ and $y$ axes is not diffeomorphic to a line.
\end{example}
\begin{proof}
  Let $A = V(xy)$ be the union of the axes.
  Suppose there was some diffeomorphism $f \from \RR \to A$.
  Since $f$ is continuous and bijective, let $I = f\inv(I')$ and $J = f\inv(J')$ where
  \[ I' = \set{ (x, y) \in A \mid x \in (-1, 1) } \]
  and
  \[ J' = \set{ (x, y) \in A \mid y \in (-1, 1) }. \]
  Because $f$ is $C^{1}$, we know both $I$ and $J$ are open.
  Clearly, $I' \cap J' = \set{0}$. Therefore, $I \cap J$ must not be empty as well.
  However, intersection of open sets is also open and so there must be more than one
  point common to the two sets.
  This contradicts the injective nature of $f$ and so they cannot be diffeomorphic.
\end{proof}

%%%%%%%%%%%

\begin{definition}{Manifold Prototype II}{manifold-proto-ii}
  A 2-manifold is a set $X \subset \RR[n]$ which is locally defined by
  an injective, regular and smooth parametrized surface.

  \[ \forall p \in X, \exists U \subset \RR[2], \alpha \from U \to V \cap X \]
  with $U$ open and connected, and $V$ open and connected in $\RR[n]$.
\end{definition}

\begin{example}{Exercise}{}
  Define  $T^{2} = F\inv(0)$.
  \[ F = \pars{\sqrt{x^{2} + y^{2}} - R}^{2} + z^{2} - r^{2}. \]
  With $R, r > 0$.

  Find local charts which cover $T^{2}$, find the minimum needed,
  and prove that $T^{2}$ cannot be globally parametrized.
\end{example}

%%%%%%%%%%%%

\begin{definition}{Manifold Without Boundary}{manifold-no-bd}
  Let $k > 0$. A $k$-manifold of class $C^{r}$ is a subset $M \subseteq \RR[n]$ (with induced topology)
  such that for all $p \in M$ there is an open $V \subseteq M$ containing $p$, an open subset $U \subseteq \RR[k]$ and a map
  \[ \alpha \from U \to V \]
  so that
  \begin{itemize}
    \item $\alpha$ is a diffeomorphism
    \item $\alpha$ is $C^{r}$
    \item $D\alpha(x)$ has rank $k$ for all $x \in U$.
  \end{itemize}
\end{definition}

\textbf{Remark}:
  When $r = 0$, they are called \emph{topological manifolds}.
  We take $r \ge 1$, and eventually will get to $r = \infty$.
  In fact when $k = 0$, then $M$ is a set of discrete points.

The pair $(\alpha, U)$ is sometimes called a \emph{coordinate patch} or \emph{chart}.
In fact, there are a couple more definitions (with $k, M$ defined as normal)
\begin{enumerate}[start=1,label={\arabic*\rparen}]
  \item[Explicit, 1\rparen]
        $\forall p \in M$, there is a neighborhood $W \subset M$ of $p$ so that $W$ is the graph
        of a $C^{r}$ function $f \from V \to \RR[n - k]$ where $V \subset \RR[k]$ is open.
        That is,
        \[ W = \set*{ \binom{x}{f(x)} \Big| x \in \RR[k], f(x) \in \RR[n - k] }. \]

  \item[Implicit, 2\rparen]
        $\forall p \in M$ there is a neighborhood $W \subset \RR[n]$ of $p$ and
        a $C^{r}$ function $F \from W \to \RR[n - k]$, such that $F\inv(0) = W \cap M$
        and $\rank DF = n - k$.
\end{enumerate}

\begin{theorem}{}{manifolds-equiv}
  The three definitions of a manifold given before are equivalent,
  with the first referred to as ``Parametric''.
\end{theorem}
\begin{proof}
  (Implicit $\implies$ Explicit)
  Let $M$ be an implicit manifold.
  Write $F = F(\vec{x}, \vec{y})$ with $\vec{x} \in \RR[k], \vec{y} \in \RR[n - k]$
  and without loss of generality (automorphism of $\RR[n]$) we may assume
  \[ \det \frac{\partial F}{\partial \vec{y}}(p) \ne 0. \]
  By the implicit function theorem, there is a $C^{R}$ function $\phi \from U \to V$
  (with $U, V$ a neighborhood of the correct parts of $p$)
  so that $\phi(\vec{x}) = \vec{y}$ and $F(\vec{x}, \phi(\vec{x})) = 0$ on all of $U$.

  Shrinking $W$ if needed, we know
  \[ (\vec{x}, \vec{y}) \in M \cap W \iff \vec{y} = \phi(\vec{x}). \]
  This means that $M \cap W$ is in fact the $W$ from the explicit case.

  (Explicit $\implies$ Parametric)
  Let $M$ be an explicit manifold given by $W, V, f$.
  Then by the inverse function theorem we may define an appropriate $\alpha$ so that
  \[ \alpha(x) = \binom{x}{f(x)}. \]
  We know $\alpha$ is bijective, and is bicontinuous by the inverse function theorem
  and the fact that
  \[ D\alpha = \begin{bmatrix} I_{k} & 0\\ 0 & Df \end{bmatrix} \]
  which has rank $k$.

  (Parametric $\implies$ Implicit)
  Let $p \in M$. Construct $\alpha \from U \to M$ a diffeomorphism with its image.
  Further, by automorphisms of $\RR[k]$ we may assume $\alpha(0) = p$.
\end{proof}

\begin{example}{}{}
  Let $\alpha(t) = (t + t^{2} \ t^{2} \ t^{3})$.
  \begin{enumerate}[start=1,label={\arabic*\rparen}]
    \item Check $\alpha$ gives a 1-manifold, where $\alpha(\RR) = M$.
    \item Find nbhd of zero so that $M \cap W$ is the zero set $F(x) = 0$.
    \item Is this gobally true (no, explain). Find the others.
  \end{enumerate}
\end{example}

\subsection{Manifolds with Boundary}

\begin{definition}{}{differentiable-on-arb-set}
  Let $S \subset \RR[k]$ be and $f \from S \to \RR[n]$.
  We say $f$ is $C^{r}$ if $f$ can be extended to a $C^{r}$ function on an open set $U \supset S$.
\end{definition}

\begin{lemma}{}{}
  Let $S \subset \RR[k]$, $f \from S \to \RR[n]$.
  If for each $x \in S$ we have an open $U_{x}$ and a $C^{r}$ function $g_{x} \from U_{x} \to \RR[n]$
  such that $g_{x}|_{S \cap U_{x}} = f|_{S \cap U_{x}}$, then $f$ is $C^{r}$.
\end{lemma}
\begin{proof}
  Cover $S$ with our $\set{U_{x}}$. Let $A = \bigcup U_{x}$ and choose $\set{\phi_{i}}$
  be a partition of unity for $A$ subordinate to $\set{U_{x}}$.
  Set $U_{i}$ to be one of the $U_{x} \supset \Support \phi_{i}$.
  Take any $C^{r}$ function $g_{i} \from U_{i} \to \RR[n]$,
  and extend it to $h_{i} \from A \to \RR[n]$ by letting $h_{i} = \phi_{i}g_{i}$ which is still $C^{r}$.

  Define
  \[ g(x) = \sum_{i = 1}^{\infty}h_{i}(x) \]
  which is in fact $C^{r}$ on all of $A$.
  Further,
  \[ g(x) = \sum \phi_{i}g_{i} = \sum \phi_{i}f = f \sum \phi_{i} = f \]
  when restricted to $S$.
\end{proof}

\begin{definition}{}{}
  Define
  \[ \mathbb{H}^{k} = \set{ x \in \RR[k] \mid x_{k} \ge 0 } \]
  and
  \[ \mathbb{H}_{+}^{k} = \set{ x \in \RR[k] \mid x_{k} > 0 }. \]
  These are the ``upper half-planes'' of $\RR[k]$ both open and closed.
\end{definition}

\begin{definition}{Ultimate Manifold (Final Final Ver3)}{}
  Let $k > 0$.
  A $k$-manifold $M$ of class $C^{r}$ is a topological subspace of $\RR[n]$ such that
  \begin{enumerate}[start=1,label={\arabic*\rparen}]
    \item For any $p \in M$, there is an open neighborhood $V \subset M$,
          an open $U$ in either $\RR[k]$ \emph{or} $\mathbb{H}^{k}$
          and a bijective map $\alpha \from U \to V$.
    \item $\alpha$ is $C^{r}$.
    \item $\alpha\inv$ is continuous.
    \item $D\alpha(x)$ has rank $k$ for all $x \in U$.
  \end{enumerate}
\end{definition}

\begin{lemma}{}{}
  Let $M \subset \RR[n]$ and $(\alpha, U)$ be a chart.
  If $U_{0} \subset U$ is open, then $\alpha|_{U_{0}} \from U_{0} \to \alpha(U_{0}) \subset V$ is also a chart.
\end{lemma}

\begin{theorem}{}{}
  Let $M \subset \RR[n]$ be a $k$-manifold of class $C^{r}$,
  and $(\alpha, U), (\beta, V)$ so that $\alpha(U) \cap \beta(V) \ne \emptyset$.
  We can define an isomorphism between $U, V$ by the composition $\alpha \circ \beta\inv$.
\end{theorem}

\begin{definition}{Pre-Manifold}{}
  A pre-manifold $M$ is a topological space with a cover of charts
  \[ \alpha \from U \subset \RR[n] \to M \]
  which are locally homeomorphic, with $C^{r}$ transition functions.
  Assume $M$ is Hausdorff and second countable.
  The $n$ in $\RR[n] \supset U$ is the dimension of $M$.
\end{definition}

\begin{theorem}{Whitney's embedding theorem}{whitney}
  An abstract manifold of dimension $n$ can be embedded
  in $\RR[2n]$, where $2n$ is a sharp bound.
\end{theorem}

There are specific words to describe the relationship between
the class of transition functions and their manifolds.
I will not list them here.

\begin{definition}{Boundary of a Manifold}{mfld-boundary}
  Let $M \subset \RR[n]$ be a $k$-manifold and $p \in M$.
  $p$ is interior if there is a chart $\alpha \from U \to V$ with $p \in V$
  so that $U$ is open in $\RR[k]$.
  Otherwise, $p$ is a boundary point.
  We denote the set as
  \[ \partial M = \set*{ p \in M \mid p \text{ not interior to } M } \]
\end{definition}

If we disregard the subspace topology, $\Bd S^{1} = S^{1}$ and $\partial S = \emptyset$.
However they are the same if we take the subspace topology.

\begin{proposition}{}{}
  $M \subset \RR[n]$ is a $k$-manifold.
  If $\partial M \ne \emptyset$, then $\partial M$ is a $(k - 1)$-manifold
  without boundary. Both manifolds are of class $C^{r}$.
\end{proposition}
\begin{proof}
  Let $p \in \partial M$ and choose a chart $\alpha \from U \to V$, which means
  $U$ is open in $\mathbb{H}^{k}$ and not open in $\RR[k]$.
  Then $\exists x \in \partial\mathbb{H}^{k}$ so that $p = \alpha(x)$.
  Note that $\alpha(U \cap \mathbb{H}_{+}^{k})$ are interior to $M$,
  and so $\alpha(U \cap \partial \mathbb{H}^{k})$ are all boundary to $M$.
  Thus $\alpha\from U \cap \partial \mathbb{H}^{k} \to V \cap \partial M$ is a bijection.

  Set $U_{0} \subset \RR[k - 1]$ to be the open set so that
  \[ U_{0} \times \set{0} = U \cap \partial \mathbb{H}^{k} \]
  with $x \in U_{0}$ and $\alpha(x) = \alpha_{0}(x, 0)$.
  We show that $\alpha_{0}$ is a chart.
  \begin{itemize}
    \item $\alpha_{0}$ is a restriction and so is $C^{r}$.
    \item $D\alpha_{0}$ has rank $k - 1$.
    \item $\alpha_{0}\inv$ is simply $\pi \circ \alpha\inv$
          with $\pi \from U \to \RR[k - 1]$ being a projection map.\ \qedhere
  \end{itemize}
\end{proof}

\begin{theorem}{}{}
  Let $U \subset \RR[n]$ be open and $f \from U \to \RR$ be of class $C^{r}$.
  Set $M = f\inv(\set{0})$ and $N = f\inv(\lbrack0, \infty\rparen)$.
  If $M \ne \emptyset$ and $\rank Df = 1$ on $M$, then
  $N$ is an $n$-manifold and $\partial N = M$.
\end{theorem}
\begin{proof}
  Pick $p \in N$ so that $f(p) > 0$ (i.e., $p$ is interior to $N$).
  Let $U = \set{ x \in \RR[n] \mid f(x) > 0 }$.
  Thus $\id[U]$ is a chart of the nonzero points of $N$.
  If $f(p) = 0$, then since $\rank Df = 1$ at these points, we assume
  without loss of generality that $D_{n}f(p) \ne 0$.

  Define $F \from U \to \RR[n]$ by
  \[ (x_{1}, \ldots, x_{n}) \to (x_{1}, \ldots, x_{n - 1}, f(x)). \]
  Therefore,
  \[ DF = \begin{bmatrix} I_{n - 1} & 0\\
    0 & D_{n}f \end{bmatrix}. \]
  Which has full rank and is thus invertible at $p$.
  Therefore, we can get $A, B \subset \RR[n]$ and $F\inv$ is a diffeomorphism
  with
  \[ F|_{A \cap M} \from A \cap M \to B \cap \partial \mathbb{H}^{n}. \qedhere \]
\end{proof}

\begin{corollary}{}{}
  The $n$-ball $B^{n}$ is a manifold with boundary $S^{n}$.
\end{corollary}

\begin{example}{Exercise}{}
  $\GL(n, \RR)$, $\SL(n, \RR)$ defined as usual.
  Show they are both manifolds.

  Hint: $\dim \GL_{n} = n^{2}$ and $\dim \SL_{n} = n^{2} - 1$.

  Bonus: Try to check multiplication and inversion are smooth.
  These are examples of \emph{real Lie groups}.
\end{example}

\section{Tangent Spaces}

\subsection{Tangent Space to $\RR[n]$}

\begin{definition}{Tangent Space of $\RR[n]$}{tgsp-Rn}
  Fix some $p \in \RR[n]$.
  The \emph{tangent space} to $\RR[n]$ at $p$ is
  the vector space $T_{p}\RR[n]$ of pairs $(p, \vec{v})$
  with $\vec{v} \in \RR[n]$.

  The addition is defined as
  \[ (p, \vec{v}) + (p, \vec{w}) = (p, \vec{v} + \vec{w}). \]
  Scalar multiplication is also second-slot only, so
  \[ \lambda(p, \vec{v}) = (p, \lambda\vec{v}). \]
\end{definition}

In $\RR[n]$ this looks like a translation so the origin is moved to $p$.

Suppose we now have $\gamma(t) \from I \to \RR[n]$ as a $C^{r}$ parametrized curve.
So, the tangent vector to $\gamma(I)$ at $\gamma(t')$ is $(\gamma(t'); D\gamma(t'))$.

\begin{definition}{}{differential}
  Let $U \subset \RR[k]$ be open (in $\RR[k]$ or $\mathbb{H}^{k}$)
  and $\alpha \from U \to \RR[n]$ be $C^{r}$.
  If $x \in U$ and $p = \alpha(x)$, then
  \[ \alpha_{*} \from T_{x}\RR[k] \to T_{p}\RR[n] \]
  where
  \[ (x, \vec{v}) \mapsto (\alpha(x), D\alpha(x) \cdot \vec{v}). \]
  This $\alpha_{*}$ is sometimes called the \emph{pushforward} or
  \emph{differential} of $\alpha$.
\end{definition}

If $A \xrightarrow{\alpha} B$ and $B \xrightarrow{\beta} \RR[n]$,
then $A \xrightarrow{\beta \circ \alpha} \RR[n]$ exists.
Then, $T_{p}A \xrightarrow{\alpha_{*}}$

\begin{definition}{Tangent Space}{tgt-space}
  $M \subset \RR[n]$ will be a $C^{r}$ $k$-manifold.
\end{definition}

\begin{example}{Exercise (2 POINTS)}{}
  Compute explicit equations for the tangent space at either:
  \begin{itemize}
    \item two distinct points with different charts.
    \item the same point with two different charts.
          Then, give the inverse $D(\alpha_{2}\inv \circ \alpha_{1})$ explicitly.
  \end{itemize}

  For the surfaces:
  \begin{itemize}
    \item $\displaystyle \pars*{\frac{x}{a}}^{2} + \pars*{\frac{y}{b}}^{2} + \pars*{\frac{z}{c}}^{2} = 1$.
    \item $\displaystyle \pars*{\frac{x}{a}}^{2} + \pars*{\frac{y}{b}}^{2} - z = 0$.
    \item $\displaystyle \pars*{\frac{x}{a}}^{2} - \pars*{\frac{y}{b}}^{2} - z = 0$.
    \item $\displaystyle \pars*{\frac{x}{a}}^{2} + \pars*{\frac{y}{b}}^{2} - \pars*{\frac{z}{c}}^{2} = \pm 1$.
  \end{itemize}
\end{example}

To get the transition functions from two charts explicitly, let us take $\alpha_{1}, \alpha_{2} \from \RR[2] \to S^{2}$ so that $\alpha_{1}$ parametrizes the $z$ coordinate, and $\alpha_{2}$ the $y$.
Clearly, the inverses of both are the projections of the corresponding coordinates; the third isn't messing with them at all.
Therefore,
\[ (\alpha_{2}\inv \circ \alpha_{1}) \begin{pmatrix} x \\ y \end{pmatrix}
  = \alpha_{2}\inv \begin{pmatrix} x \\ y \\ \sqrt{1 - x^{2} - y^{2}} \end{pmatrix}
  = \begin{pmatrix} x \\ \sqrt{1 - x^{2} - y^{2}}. \end{pmatrix}
\]

\begin{theorem}{}{}
  Let $M \subset \RR[n]$ be implicitly defined by $F = 0$ locally.
  For $p \in W \subset M$, $W$ must be described by $F = 0$.
  Then,
  \[ T_{p}M = \ker DF(p). \]
\end{theorem}

\begin{example}{}{}
  Compute the volume formula for a regular parametrized 3-manifold living in $\RR[4]$. That is,
  \[ M = \alpha(\RR[3]) \]
  and
  \[ \alpha(x, y, z) = (x, y, z, f(x, y, z)). \]
\end{example}

\section{Differential Forms}

A differential 1-form is a smooth function from $\M \to T^{\ast}\M$, or in a specific case $\RR[n] \to (p, \pars{T_{p}\RR[n]}^{\ast})$.

\begin{definition}{Differential $k$-form}{}
  A differential $k$-form at $p \in \M$ is a function
  \[
    \omega \from \underbrace{T_{p}\M \times \cdots T_{p}\M}_{k \text{ times }} \to \RR
  \]
  where $\omega$ is both multilinear and alternating.
  We call this vector space of functions $A^{k}(V)$ where $V$ is the underlying vector space (above, $T_{p}\M$.)
\end{definition}

\begin{example}{Differental forms}{}
  A 2-form on $\RR[3]$ is an element in $A^{2}(\RR[3])$. For instance,
  \[ \Theta_{(1, 2)}(\vec{v}, \vec{w}) = \det \begin{bmatrix} v_{1} & w_{1} \\ v_{2} & w_{2} \end{bmatrix} \]
  In general for an index set $I \subset \set{1, \ldots, n}$ we can pick out $\abs{I}$ rows of the $n \times \abs{I}$ matrix.
  So, we can craft (where $\abs{I} = k \le n$.)
  \[ \Theta_{I} \from \underbrace{\RR[n] \times \cdots \times \RR[n]}_{k \text{ times }} \to M_{k \times k} \]
\end{example}

\begin{proposition}{}{}
  Given any $\phi \in A^{k}(\RR[n])$, we may write
  \[ \phi = \sum_{I} a_{I} \Theta_{I} \]
  where $a_{I}$ are unique numbers.
\end{proposition}

\begin{example}{Exercise}{}
  Prove that
  \[ \sum_{\sigma \in S_{k}} (\operatorname{sgn} \sigma) v_{i_{1}, \sigma(1)} \cdot \cdots \cdot v_{i_{k}, \sigma(k)} = \Theta_{(i_{1}, \ldots, i_{k})}(v_{1}, \ldots, v_{k}) \]
  It will be easier to show this for $n \times n$ matrices and $\det$ (where you use $\det [v_{1}, \ldots v_{k}]$ with $k$ rows chosen).
\end{example}

\begin{definition}{Wedge Product}{wedge-prod}
  Let $\Theta_{I} \in A^{k}(\RR[n]), \Theta_{J} \in A^{\ell}(\RR[n])$. Then,
  \[ \Theta_{I} \wedge \Theta_{J} = \Theta_{(I, J)}. \]
  That is, $\wedge \from A^{k}(V) \times A^{\ell}(V) \to A^{k + \ell}(V)$.
\end{definition}
\begin{example}{}{}
  Define
  \begin{align*}
    \omega = 5\Theta_{1} - 2\Theta_{2} + \Theta_{3}\\
    \eta = 2\Theta_{(1, 2)} + \Theta_{(2, 3)}.
  \end{align*}
  Then,
  \[ \omega \wedge \eta =
    10(\Theta_{1} \wedge \Theta_{(1, 2)}) + 5(\Theta_{1} \wedge \Theta_{(2, 3)}) - 4(\Theta_{2} \wedge \Theta_{(1, 2)})
    - 2(\Theta_{2} \wedge \Theta_{(2, 3)}) + 2(\Theta_{3} \wedge \Theta_{(1, 2)}) + \Theta_{3} \wedge \Theta_{(2, 3)}.
  \]
  This can be collapsed, as $\Theta_{I} \wedge \Theta_{J} = 0$ when $I$ and $J$ are not disjoint.
  We get $5\Theta_{(1, 2, 3)} + 2\Theta_{(3, 1, 2)}$ and by transposing indices and keeping track of the sign, the final answer is $7\Theta_{(1, 2, 3)}$.
\end{example}

\begin{proposition}{}{}
  The wedge product is
  \begin{enumerate}[start=1,label={\arabic*\rparen}]
    \item bilinear: $(\omega + \phi) \wedge \eta = \omega \wedge \eta + \phi \wedge \eta$ (with $\omega, \phi$ in the same space).
    \item skew-commutative: $\omega \wedge \eta = \pars{-1}^{k\ell}(\eta \wedge \omega)$
    \item associative.
  \end{enumerate}
\end{proposition}

Globalizing this to manifolds is the next step.
We define
\[ \Omega^{k}(U) = \set*{ \mbox{\text{ differentiable $k$-forms on $U$ }} } \]

\end{document}
