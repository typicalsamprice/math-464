% -*- compile-command: "latexmk -pdf document.tex" -*-
\documentclass{article}

\author{Sam Price}
\title{Capstone: Inverse Function Theorem}

\newif\ifprinted%
\printedtrue%
% Most taken from: https://github.com/SeniorMars/dotfiles/blob/main/latex_template/preamble.tex
\usepackage{geometry}

\usepackage[english]{babel}
\usepackage[T1]{fontenc}
\usepackage[utf8]{inputenc}
\usepackage{palatino}

% Use the command \doublespacing if needed
\usepackage{setspace}

\geometry{a4paper, margin=1in}

\usepackage{mathtools}
\usepackage{amssymb,amsfonts,amsthm,amsmath}

\usepackage{xfrac}
\usepackage[makeroom]{cancel}

% Use \begin{enumerate}[start=x,label={Q\arabic*)}] for example
\usepackage{enumitem}

\usepackage{xcolor}

\usepackage{nameref}

% Important options
% in envs, use options like [baseline=x] to center on row x (or special opts t/[c]/b without 'baseline')
\usepackage{nicematrix}
\NiceMatrixOptions{cell-space-limits = 1pt}

\usepackage{booktabs}

\usepackage{tikz}
\usepackage{tikz-cd}
\usepackage{tikzsymbols}

\usepackage{pdfpages}

\usepackage[most,many,breakable]{tcolorbox}

\setlength{\parindent}{1cm}

\DeclarePairedDelimiter{\abs}{\lvert}{\rvert}

%%%%%%%%%%%%%%%%%%%%%
%%% THEOREM BOXES %%%
%%%%%%%%%%%%%%%%%%%%%
\definecolor{thmbgcol}{HTML}{aec1f9}
\definecolor{thmhlcol}{HTML}{142c72}
\definecolor{corbgcol}{HTML}{b599f7}
\definecolor{corhlcol}{HTML}{2d1760}
\definecolor{propbgcol}{HTML}{c9f7aa}
\definecolor{prophlcol}{HTML}{1e631a}
\definecolor{exbgcol}{HTML}{f7c479}
\definecolor{exhlcol}{HTML}{604419}

\definecolor{qheadcol}{HTML}{182959}

\tcbuselibrary{theorems,skins,hooks}
\newtcbtheorem[number within = section]{theorem}{Theorem}{
  enhanced, breakable, colback = thmbgcol!25,
  frame hidden, boxrule = 0sp, borderline west = {2pt}{0pt}{thmhlcol},
  sharp corners, detach title, before upper = \tcbtitle\par\smallskip,
  coltitle = thmhlcol, fonttitle = \bfseries,
  description font = \mdseries, separator sign none, segmentation style = {solid, thmhlcol}
}{th}

\newtcbtheorem[number within = section]{corollary}{Corollary}{
  enhanced, breakable, colback = corbgcol!25,
  frame hidden, boxrule = 0sp, borderline west = {2pt}{0pt}{corhlcol},
  sharp corners, detach title, before upper = \tcbtitle\par\smallskip,
  coltitle = corhlcol, fonttitle = \bfseries,
  description font = \mdseries, separator sign none, segmentation style = {solid, corhlcol}
}{cor}

\newtcbtheorem[number within = section]{proposition}{Proposition}{
  enhanced, breakable, colback = propbgcol!25,
  frame hidden, boxrule = 0sp, borderline west = {2pt}{0pt}{prophlcol},
  sharp corners, detach title, before upper = \tcbtitle\par\smallskip,
  coltitle = prophlcol, fonttitle = \bfseries,
  description font = \mdseries, separator sign none, segmentation style = {solid, prophlcol}
}{prop}

\newtcbtheorem[number within = section]{example}{Example}{
  enhanced, breakable, colback = exbgcol!25,
  frame hidden, boxrule = 0sp, borderline west = {2pt}{0pt}{exhlcol},
  sharp corners, detach title, before upper = \tcbtitle\par\smallskip,
  coltitle = exhlcol, fonttitle = \bfseries,
  description font = \mdseries, separator sign none, segmentation style = {solid, exhlcol}
}{prop}

\newtcbtheorem[number within = section]{definition}{Definition}{
  enhanced, breakable, colback = red!10,
  frame hidden, boxrule = 0sp, borderline west = {2pt}{0pt}{red!50!black},
  sharp corners, detach title, before upper = \tcbtitle\par\smallskip,
  coltitle = red!50!black, fonttitle = \bfseries,
  description font = \mdseries, separator sign none, segmentation style = {solid, exhlcol}
}{def}

\makeatletter
\newtcbtheorem{question}{Question}{enhanced,
    breakable,
    colback=white,
    colframe=qheadcol,
    attach boxed title to top left={yshift*=-\tcboxedtitleheight},
    fonttitle=\bfseries,
    title={#2},
    boxed title size=title,
    boxed title style={%
            sharp corners,
            rounded corners=northwest,
            colback=qheadcol,
            boxrule=0pt,
        },
    underlay boxed title={%
            \path[fill=tcbcolframe] (title.south west)--(title.south east)
            to[out=0, in=180] ([xshift=5mm]title.east)--
            (title.center-|frame.east)
            [rounded corners=\kvtcb@arc] |-
            (frame.north) -| cycle;
        },
    #1
}{def}
\makeatother


\begin{document}

\maketitle

\section{Pre-IFT Work}

This is normally done on any arbitrary complete metric space $X$, but I'd rather not also prove that contraction maps are necessarily continuous.
\begin{theorem}{Banach Fixed-Point Theorem}{banach}
  Let $f \from A \to A$ be a continuous\footnote{They are necessarily continuous, but I will require it for simplicity}
  function on a closed subset $A \subset \RR[n]$ so that there is a fixed $q \in \lbrack 0, 1 \rparen$:
  \[ \norm{f(x) - f(y)} \le q\norm{x - y} \]
  for all $x, y \in \RR[n]$.
  Then, there is a unique fixed point $x^{*}$.
\end{theorem}
\begin{proof}
  Let $x_{0} \in A$ and define the sequence $\set{x_{n}}$ by $x_{n + 1} = f(x_{n})$.
  We notice that
  \[ \norm{x_{2} - x_{3}} \le q\norm{x_{1} - x_{2}} \le q^{2}\norm{x_{0} - x_{1}}. \]
  Extrapolating this, we notice
  \[ \norm{x_{n} - x_{n + 1}} \le q^{n}\norm{x_{0} - x_{1}}. \]
  Let $\eps > 0$ now, and pick $N$ large enough so that
  \[ q^{N} < \frac{\eps(1 - q)}{\norm{x_{0} - x_{1}}}. \]
  For $m > n > N$, we find that
  \begin{align*}
    \norm{x_{n} - x_{m}} &\le \norm{x_{m} - x_{m - 1}} + \norm{x_{m - 1} - x_{m - 2}} + \cdots + \norm{x_{n + 1} - x_{n}}\\
    &\le q^{m - 1}\norm{x_{0} - x_{1}} + q^{m - 2}\norm{x_{0} - x_{1}} + \cdots + q^{n}\norm{x_{0} - x_{1}}\\
    &= q^{n}\norm{x_{0} - x_{1}} \sum_{k = 0}^{m - n - 1}q^{k}\\
    &\le q^{n}\norm{x_{0} - x_{1}} \sum_{k = 0}^{\infty}q^{k}.
  \end{align*}
  The reason this is not a \emph{strict} inequality is that $q = 0$ is permitted.
  Then, we find that
  \[
    q^{n}\norm{x_{0} - x_{1}} \sum_{k = 0}^{\infty}q^{k} = \frac{q^{n}}{1 - q}\norm{x_{0} - x_{1}}.
  \]
  Since $n > N$, we know $q^{n} \le q^{N}$. This tells us the right hand side has:
  \[ \frac{q^{n}}{1 - q}\norm{x_{0} - x_{1}} < \frac{\eps(1 - q)}{(1 - q)\norm{x_{0} - x_{1}}}\norm{x_{0} - x_{1}} = \eps. \]
  Because our sequence $\set{x_{n}}$ is Cauchy and $\RR[n]$ is complete, as proven by yours truly, we know that $\set{x_{n}} \to x^{*}$ for some $x^{*}$.
  Suppose then there are two fixed points of $f$: $a, b \in A$. We find that
  \[ \norm{a - b} > \norm{f(a) - f(b)} = \norm{a - b}. \]
  This isn't possible though if $a \ne b$ and so we reach a contradiction, showing $x^{*}$ is unique.

  The reason $A$ must be closed is so that every limit in $A$ converges to something inside $A$.
  An open set might not have this property, and so we require it.
\end{proof}

\begin{lemma}{}{tao-contraction}
  Let $g \from B_{r}(0) \to \RR[n]$ be a contraction map with constant $q = 1/2$ so that $g(0) = 0$.
  Then, $f(x) = g(x) + x$ is injective and
  \[ B_{r/2}(0) \subset f(B_{r}(0)). \]
\end{lemma}
\begin{proof}
  Suppose first there are $x, y$ so $f(x) = f(y)$. This means that
  \begin{align*}
    x + g(x) &= y + g(y)\\
    \implies \abs{g(x) - g(y)} &= \abs{y - x}.
  \end{align*}
  This means that $y - x = 0$, and so $y = x$. Thus, $f$ is injective.

  Now, let $y \in B_{r/2}(0)$. We must now find some $x \in B_{r}(0)$ so that $f(x) = y$.
  This is equivalent to finding $x$ so that $x = y - g(x)$, or $x$ being a fixed point of $x \mapsto y - g(x)$.
  Call this map new map $F$, so $F(x) = y - g(x)$. Note for $x \in B_{r}(0)$ there is
  \[ \abs{F(x)} \le \abs{y} + \abs{g(x)} \le \frac{r}{2} + \abs{g(x) - g(0)} \le \frac{r}{2} + \frac{1}{2}\abs{x - 0} < \frac{r}{2} + \frac{r}{2} < r. \]
  So, $F(B_{r}) \subset B_{r}$.
  We also have the inequality then for $x, x' \in B_{r}(0)$ that
  \begin{equation}\label{eq:f-strict-contraction} \abs{F(x) - F(x')} = \abs{g(x) - g(x')} \le \frac{1}{2}\abs{x - x'}. \end{equation}

  In fact, for a sufficiently small $\eps > 0$ we can have $r - \eps = s$ and
  $f(\overline{B_{s}}) \subset \overline{B_{s}}$ is mapped to itself\footnote{Since we would end with $\abs{F(x)} < r - \eps$, by looking at $B_{s/2}$.},
  since our inequality is strict there is some wiggle room.
  Because $\overline{B_{s}}$ is complete, we may apply the Banach fixed-point theorem to $F$ by~\eqref{eq:f-strict-contraction},
  and so our satisfactory $x$ so $f(x) = y$ is found. Therefore, $f(B_{r}) \supset B_{r/2}$.
\end{proof}

\begin{lemma}{[Skip if comfortable]}{linear-inversion-is-linear}
  If $T \from \RR[n] \to \RR[n]$ is a linear transformation that is invertible,
  then $T\inv$ is also linear.
\end{lemma}
\begin{proof}
  We know that by taking $I = T\inv \circ T$, there is
  \[ T\inv(T(T\inv(x) + T\inv(y))) = T\inv(T(T\inv x) + T(T\inv y))
    = T\inv(x + y). \]
  Further,
  \[ a T\inv x = T\inv(T(a T\inv x)) = T\inv(a x). \]
  So, $T\inv$ is linear.
\end{proof}

\section{The Inverse Function Theorem}

\begin{theorem}{Inverse Function Theorem}{}
  Let $f \from A \to \RR[n]$ with $A \subset \RR[n]$ open and $f \in C^{r}$.
  If $x_{0} \in A$ has the property $Df(x_{0})$ is invertible, then there is some open $U \subset A$ about $x_{0}$ and $V \subset \RR[n]$ about $f(x_{0})$ so that
  $f|_{U} \from U \to V$ is a bijection and the inverse map $f \inv \from V \to U$ exists and is differentiable at $x_{0}$. Further,
  \[ Df\inv(f(x_{0})) = Df(x_{0})\inv. \]
\end{theorem}
\begin{proof}
  We show that we only need to work in a special case where $x_{0} = 0$, $f(0) = 0$ and $Df(0) = I_{n}$.
  Firstly, suppose $f(x_{0}) = 0$. If not, then create $\tilde{f}(x) = f(x) - f(x_{0})$.
  We see that $D\tilde{f}(x_{0}) = Df(x_{0})$, and $\tilde{f}\inv(y - f(x_{0})) = f\inv(y)$ which means $D\tilde{f}\inv(0) = Df\inv(f(x_{0}))$, as desired.

  Then, we may assume $x_{0} = 0$ by a similar construction. Take $\tilde{f}(x) = f(x + x_{0})$, and we see $D\tilde{f}(0) = Df(x_{0})$ and $f\inv(y) = \tilde{f}\inv(y) + x_{0}$
  to arrive at $D\tilde{f}\inv(0) = Df\inv(0)$. Then
  \[ D\tilde{f}\inv(0) = Df\inv(0) = Df(0)\inv = D\tilde{f}(0)\inv. \]

  \newpage
  Now, we must show that the assumption $Df(0) = I_{n}$ is okay. Define the function $\tilde{f}(x) = Df(0)\inv f(x)$.
  We clearly see that $D\tilde{f}(0) = Df(0)\inv Df(0) = I_{n}$. By the special case, there are open $U' \ni 0$, $V' \ni 0$ with $\tilde{f}\inv \from V' \to U'$ differentiable at 0.
  However, $f(x) = Df(0) \tilde{f}(x)$ and so $f \from U' \to Df(0) V'$.
  Because $Df(0)$ is a bijection, $f$ must also be then.
  The inverse $f\inv(y) = \tilde{f}\inv(Df(0)\inv y)$ must then be differentiable at zero as well.

  Now we simply prove the inverse function theorem for $x_{0} = 0$, $f(0) = 0$ and $Df(0) = I$.
  Define the function $g: A \to \RR[n]$ by $g(x) = f(x) - x$. Naturally, $g(0) = 0$, $Dg(0) = 0$ and $D_{i}g(0) = 0$.
  Because $g \in C^{1}$, we may find a neighborhood $B = B_{r}(0)$ so that\footnote{We can bound each component by $1/2n$, and there are $n$ many of them.}
  \[ \abs{D_{i}g(x)} \le \frac{1}{2n^{2}} \]
  for all $x \in B$.
  Taking the directional derivative on $B$ for $v = (v_{1}, \ldots, v_{n})$:
  \begin{align*}
    \abs{g'(x; v)} &= \abs{Dg(x)v}\\
    &= \abs*{\sum_{i = 1}^{n}v_{i} \cdot D_{i}g(x)}\\
    &= \sum_{i = 1}^{n}\abs{v_{i}} \abs{D_{i}g(x)}\\
    &\le \sum_{i = 1}^{n}\frac{\abs{v}}{2n^{2}} = \frac{v}{2n}.
  \end{align*}
  By the fundamental theorem of calculus, we find that for $x, y \in B$
  \[ g(y) - g(x) = \int_{0}^{1}\frac{d}{dt}g(x + t(y - x)) \]
  which can be written on the left side as
  \[ \int_{0}^{1} g'(x + t(y - x); y - x). \]
  Since $Dg(x + t(y - x)) \cdot (y - x)$ gives a vector with entries $D_{i}g(x + t(y - x)) \cdot (y - x)$.

  Each of these entries has magnitude at most $\frac{1}{2n} \abs{y - x}$ by our previous bound.
  Chasing this bound around, it gives us that $g(y) - g(x)$ has components with the same bound of magnitude, and so we get
  \[ \abs{g(y) - g(x)} \le \frac{1}{2}\abs{y - x}. \]
  This means that $g$ is a contraction map with coefficient $1/2$, and so can be used in the previous lemma.
  This means that our original $f(x) = g(x) + x$ is injective on some ball $B_{r}(0)$ whose image contains $B_{r/2}(0)$.
  Restricting our $f$ to $f\inv(B_{r/2}(0)) \cap B_{r}(0)$, we get a function that is bijective around 0, and so an inverse function is well-defined.

  Using the bounds of contraction with $g$, we see for $x \in B_{r}(0)$:
  \begin{align}
    \label{eq:contraction-bounds}
    \begin{split}
      \abs{g(x) - g(0)} &= \abs{g(x)} \le \frac{1}{2}\abs{x}\\
      \implies
      \frac{1}{2}\abs{x} &\le \abs{f(x)} \le \frac{3}{2}\abs{x}.
    \end{split}
  \end{align}

  To show $f\inv$ is differentiable at zero\footnote{Maybe replace $x$ with $h$, especially when presenting.} (with derivative $I$), we want
  \[ \lim_{x \to 0} \frac{\abs{f\inv(x) - f\inv(0) - I(x - 0)}}{\abs{x}} = 0 \]
  which because $f(0) = 0$, then $f\inv(0) = 0$, giving:
  \[ \lim_{x \to 0} \frac{\abs{f\inv(x) - x}}{\abs{x}} = 0. \]
  We will show that this is true for a sequence $\set{x_{n}} \to 0$ in $V$. Define the sequence $y_{n} = f\inv(x_{n})$.
  Using our bounds in~\eqref{eq:contraction-bounds}, we can say that $y_{n} \to 0$ as well. So, it is equivalent to then show
  \[ \lim_{n \to \infty}\frac{\abs{y_{n} - f(y_{n})}}{\abs{y_{n}}} = 0 \]
  However, $f$ is differentiable at zero, and so we \emph{know}
  \[ \lim_{n \to \infty}\frac{\abs{f(y_{n}) - f(0) - Df(0)(y_{n} - 0)}}{\abs{y_{n}}} = 0. \]
  We know that $Df(0) = I$ and $f(0) = 0$ though, and so
  \[ \lim_{n \to \infty}\frac{\abs{f(y_{n}) - y_{n}}}{\abs{y_{n}}} = 0. \]
  Thus, $Df\inv(0) = I$.
\end{proof}



\end{document}
